%% ARKHEION AGI 2.0 - PAPER TEMPLATE
%% ==================================
%%
%% Template padrão para todos os papers da árvore ARKHEION
%% Autor: Jhonatan Vieira Feitosa | Manaus-AM, Brasil
%%
%% Compilar: pdflatex paper.tex
%%

\documentclass[11pt,twocolumn]{article}

% ============================================================================
% PACKAGES
% ============================================================================
\usepackage[utf8]{inputenc}
\usepackage[T1]{fontenc}
\usepackage[english]{babel}
\usepackage{amsmath,amssymb,amsfonts}
\usepackage{graphicx}
\usepackage{booktabs}
\usepackage{hyperref}
\usepackage{xcolor}
\usepackage{listings}
\usepackage{float}
\usepackage{subcaption}
\usepackage{multirow}
\usepackage{array}
\usepackage{tikz}
\usetikzlibrary{shapes,arrows,positioning,calc,patterns,decorations.pathreplacing,fit,backgrounds}
\usepackage{pgfplots}
\pgfplotsset{compat=1.18}

% ============================================================================
% ARKHEION COLOR SCHEME
% ============================================================================
\definecolor{arkheion-blue}{RGB}{0,100,200}
\definecolor{arkheion-purple}{RGB}{128,0,128}
\definecolor{arkheion-gold}{RGB}{255,193,37}
\definecolor{arkheion-green}{RGB}{0,150,0}
\definecolor{arkheion-red}{RGB}{200,50,50}
\definecolor{code-bg}{RGB}{245,245,245}
\definecolor{code-comment}{RGB}{0,128,0}
\definecolor{code-keyword}{RGB}{0,0,200}
\definecolor{code-string}{RGB}{163,21,21}

% ============================================================================
% CODE LISTINGS
% ============================================================================
\lstset{
    basicstyle=\ttfamily\small,
    keywordstyle=\color{code-keyword}\bfseries,
    commentstyle=\color{code-comment},
    stringstyle=\color{code-string},
    backgroundcolor=\color{code-bg},
    numbers=left,
    numberstyle=\tiny\color{gray},
    breaklines=true,
    frame=single,
    framesep=3pt,
    xleftmargin=15pt,
    xrightmargin=5pt,
    tabsize=4,
    showstringspaces=false,
    captionpos=b
}

\lstdefinestyle{python}{
    language=Python,
    morekeywords={self,True,False,None,async,await,dataclass,Optional}
}

\lstdefinestyle{cpp}{
    language=C++,
    morekeywords={override,final,nullptr,constexpr,noexcept,auto}
}

% ============================================================================
% CUSTOM COMMANDS
% ============================================================================
% Mathematical symbols
\newcommand{\phi}{\varphi}
\newcommand{\Phi}{\Phi}
\newcommand{\goldenratio}{1.618033988749895}
\newcommand{\goldenangle}{137.50776405003785}

% Paper-specific
\newcommand{\arkheion}{\textsc{Arkheion}}
\newcommand{\huam}{\textsc{HUAM}}
\newcommand{\nucleus}{\textsc{Nucleus}}
\newcommand{\iit}{\textsc{IIT}}

% Boxes
\newcommand{\keypoint}[1]{\colorbox{arkheion-gold!20}{\parbox{0.9\columnwidth}{\textbf{Key Point:} #1}}}
\newcommand{\warning}[1]{\colorbox{arkheion-red!20}{\parbox{0.9\columnwidth}{\textbf{Warning:} #1}}}

% ============================================================================
% TITLE CONFIGURATION
% ============================================================================
\hypersetup{
    colorlinks=true,
    linkcolor=arkheion-blue,
    urlcolor=arkheion-purple,
    citecolor=arkheion-green,
    pdftitle={ARKHEION AGI Paper},
    pdfauthor={Jhonatan Vieira Feitosa}
}

% ============================================================================
% DOCUMENT
% ============================================================================
\begin{document}

% --- TITLE ---
\title{\textbf{[PAPER TITLE]}}

\author{
    \textbf{Jhonatan Vieira Feitosa} \\
    Independent Researcher \\
    Manaus, Amazonas, Brazil \\
    \texttt{jhonatan@arkheion.ai} \\
    \\
    \textit{ARKHEION AGI Research} \\
    \textit{Quantum Computing \& Consciousness Laboratory}
}

\date{January 2026 --- Version 1.0}

\maketitle

% --- ABSTRACT ---
\begin{abstract}
[150-250 palavras descrevendo o problema, metodologia, resultados principais e conclusões]

\textbf{Keywords:} keyword1, keyword2, keyword3, keyword4, keyword5
\end{abstract}

% ============================================================================
\section*{Epistemological Note}
% ============================================================================

\textit{This paper distinguishes between \textbf{heuristic} concepts (metaphors guiding design)
and \textbf{empirical} results (measurable outcomes). Heuristic terms like ``holographic'' or
``quantum-inspired'' serve as conceptual frameworks---visual transcriptions of the author's
mental models---not claims of literal physics. All performance claims are backed by
reproducible benchmarks.}

\vspace{0.3em}
\begin{tabular}{@{}p{0.45\columnwidth}p{0.45\columnwidth}@{}}
\textbf{Heuristic (Conceptual):} & \textbf{Empirical (Measured):} \\
\footnotesize [List metaphors used] & \footnotesize [List actual metrics] \\
\end{tabular}

% ============================================================================
\section{Introduction}
% ============================================================================

[Introdução ao problema, motivação, e contribuições do paper]

\subsection{Problem Statement}

[Qual problema estamos resolvendo?]

\subsection{Contributions}

This paper makes the following contributions:
\begin{enumerate}
    \item Contribution 1
    \item Contribution 2
    \item Contribution 3
\end{enumerate}

% ============================================================================
\section{Background}
% ============================================================================

[Teoria necessária para entender o paper]

\subsection{Theoretical Foundation}

[Matemática, física, ou teoria de CS relevante]

\subsection{Related Concepts}

[Conceitos relacionados que o leitor precisa conhecer]

% ============================================================================
\section{Methodology}
% ============================================================================

[Como o sistema funciona]

\subsection{Architecture Overview}

% Example architecture diagram
\begin{figure}[h]
\centering
\begin{tikzpicture}[
    node distance=1cm,
    box/.style={rectangle, rounded corners, draw=black, fill=#1, minimum width=2cm, minimum height=0.8cm, align=center, font=\small},
    arrow/.style={->, >=stealth, thick}
]
    \node[box=arkheion-blue!30] (input) {Input};
    \node[box=arkheion-green!30, right=of input] (process) {Process};
    \node[box=arkheion-purple!30, right=of process] (output) {Output};

    \draw[arrow] (input) -- (process);
    \draw[arrow] (process) -- (output);
\end{tikzpicture}
\caption{System architecture overview.}
\label{fig:architecture}
\end{figure}

\subsection{Algorithm}

[Descrição do algoritmo principal]

% ============================================================================
\section{Implementation}
% ============================================================================

[Detalhes de implementação]

\subsection{Code Structure}

\begin{lstlisting}[style=python, caption={Example implementation}]
class ExampleComponent:
    """Component description."""

    def __init__(self, config: dict):
        self.config = config
        self.phi = 1.618033988749895  # Golden ratio

    def process(self, data: np.ndarray) -> np.ndarray:
        """Process input data."""
        return data * self.phi
\end{lstlisting}

\subsection{Key Components}

[Descrição dos componentes principais]

% ============================================================================
\section{Experiments}
% ============================================================================

[Setup experimental e metodologia de teste]

\subsection{Experimental Setup}

\begin{itemize}
    \item Hardware: AMD Radeon RX 6600M, 8GB VRAM
    \item Software: Python 3.12, PyTorch 2.4.1+rocm6.0
    \item Dataset: [descrição do dataset]
\end{itemize}

\subsection{Metrics}

[Métricas usadas para avaliação]

% ============================================================================
\section{Results}
% ============================================================================

[Resultados experimentais]

\subsection{Quantitative Results}

% Example results table
\begin{table}[h]
\centering
\caption{Experimental Results}
\begin{tabular}{lrrr}
\toprule
\textbf{Method} & \textbf{Metric 1} & \textbf{Metric 2} & \textbf{Time} \\
\midrule
Baseline & 0.50 & 0.60 & 100ms \\
Ours & \textbf{0.85} & \textbf{0.90} & 50ms \\
\bottomrule
\end{tabular}
\label{tab:results}
\end{table}

% Example bar chart
\begin{figure}[h]
\centering
\begin{tikzpicture}
\begin{axis}[
    title={\textbf{Performance Comparison}},
    ybar,
    bar width=15pt,
    height=5cm,
    width=\columnwidth,
    symbolic x coords={Baseline,Ours},
    xtick=data,
    ymin=0,
    ymax=1,
    ylabel={Score},
    nodes near coords,
    every node near coord/.append style={font=\tiny},
]
\addplot[fill=arkheion-blue!60] coordinates {(Baseline,0.50) (Ours,0.85)};
\end{axis}
\end{tikzpicture}
\caption{Performance comparison between baseline and our method.}
\label{fig:performance}
\end{figure}

\subsection{Qualitative Analysis}

[Análise qualitativa dos resultados]

% ============================================================================
\section{Discussion}
% ============================================================================

[Discussão dos resultados, implicações, limitações]

\subsection{Analysis}

[Análise profunda dos resultados]

\subsection{Limitations}

[Limitações conhecidas]

\subsection{Future Work}

[Direções futuras]

% ============================================================================
\section{Related Work}
% ============================================================================

[Comparação com trabalhos relacionados]

\begin{itemize}
    \item \textbf{Work 1}: Description and comparison
    \item \textbf{Work 2}: Description and comparison
\end{itemize}

% ============================================================================
\section{Conclusion}
% ============================================================================

[Resumo das contribuições e conclusões]

% ============================================================================
\section*{Acknowledgments}
% ============================================================================

This work was developed as part of ARKHEION AGI 2.0, a quantum-holographic cognitive architecture system.

% ============================================================================
% REFERENCES
% ============================================================================
\begin{thebibliography}{99}

\bibitem{maldacena1999}
J. Maldacena, ``The Large N Limit of Superconformal Field Theories and Supergravity,'' \textit{Adv. Theor. Math. Phys.}, vol. 2, pp. 231-252, 1998.

\bibitem{tononi2016}
G. Tononi et al., ``Integrated Information Theory: From Consciousness to its Physical Substrate,'' \textit{Nature Reviews Neuroscience}, vol. 17, pp. 450-461, 2016.

\bibitem{nickel2017}
M. Nickel \& D. Kiela, ``Poincaré Embeddings for Learning Hierarchical Representations,'' \textit{NeurIPS}, 2017.

\end{thebibliography}

\end{document}
