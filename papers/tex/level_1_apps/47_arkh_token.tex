% ARKHEION AGI 2.0 - Paper 47: ARKH Token & Proof-of-Utility Ledger
% Jhonatan Vieira Feitosa | Manaus, Amazonas, Brazil
% February 2026

\documentclass[11pt,twocolumn]{article}

% Encoding and fonts
\usepackage[utf8]{inputenc}
\usepackage[T1]{fontenc}
\usepackage{lmodern}

% Layout
\usepackage[margin=0.75in]{geometry}
\usepackage{fancyhdr}

% Mathematics
\usepackage{amsmath,amssymb}

% Graphics and colors
\usepackage{xcolor}
\usepackage{tikz}
\usetikzlibrary{arrows.meta,shapes,positioning}

% Tables
\usepackage{booktabs}
\usepackage{multirow}

% Code listings
\usepackage{listings}

% Hyperlinks
\usepackage{hyperref}

% ==================== COLORS ====================
\definecolor{arkblue}{RGB}{0,102,204}
\definecolor{arkpurple}{RGB}{102,51,153}
\definecolor{arkgreen}{RGB}{0,153,76}
\definecolor{arkgold}{RGB}{218,165,32}

% ==================== LISTINGS ====================
\lstset{
    basicstyle=\ttfamily\scriptsize,
    breaklines=true,
    breakatwhitespace=true,
    postbreak=\mbox{\textcolor{gray}{$\hookrightarrow$}\space},
    columns=flexible,
    keepspaces=true,
    showstringspaces=false,
    numbers=none,
    backgroundcolor=\color{gray!5},
    frame=single,
    rulecolor=\color{gray!30}
}

\lstdefinestyle{python}{
    language=Python,
    morekeywords={self,True,False,None,dataclass,Optional,List,Dict}
}

% ==================== HEADER/FOOTER ====================
\pagestyle{fancy}
\fancyhf{}
\fancyhead[L]{\small\textcolor{arkblue}{ARKHEION AGI 2.0}}
\fancyhead[R]{\small Paper 47: ARKH Token \& PoU Ledger}
\fancyfoot[C]{\thepage}
\renewcommand{\headrulewidth}{0.4pt}

% ==================== HYPERREF ====================
\hypersetup{
    colorlinks=true,
    linkcolor=arkblue,
    urlcolor=arkpurple,
    citecolor=arkgreen,
    pdftitle={ARKH Token and Proof-of-Utility Ledger},
    pdfauthor={Jhonatan Vieira Feitosa}
}

% ==================== TITLE ====================
\title{
    \vspace{-1.5cm}
    {\Large\textbf{ARKH Token and Proof-of-Utility Ledger}}\\[0.3em]
    {\large A Ternary-Native Cryptocurrency for\\Computational Intelligence Markets}\\[0.2em]
    {\normalsize ARKHEION AGI 2.0 --- Paper 47}
}

\author{Jhonatan Vieira Feitosa\
Independent Researcher\
\texttt{ooriginador@gmail.com}\
Manaus, Amazonas, Brazil}

\date{February 2026}

\begin{document}

\maketitle

% ==================== ABSTRACT ====================
\begin{abstract}
We introduce the \textbf{ARKH Token}, a utility cryptocurrency built on
a \textit{Proof-of-Utility} (PoU) consensus mechanism where tokens are minted
in proportion to verified computational contributions to the \textsc{Arkheion}
AGI ecosystem. Unlike Proof-of-Work (PoW) which consumes energy for arbitrary
hash puzzles, or Proof-of-Stake (PoS) which rewards capital ownership,
PoU rewards participants for completing \textit{useful} AI computation:
neural network training, gene evolution, holographic compression, and
consciousness evaluation. The ledger is implemented as a ternary-native
blockchain (balanced ternary: $\{-1, 0, +1\}$) with 21 Python modules
totaling 13,139~lines of code. Key components include: genesis block
generation, SHA-256/Ed25519 cryptographic primitives, a PID-controlled
burn mechanism for deflationary pressure, hardware-bound wallet
identity, governance voting, transaction pool management, and persistent
block storage. The system supports 61 REST API endpoints documented
in Sphinx RST format.

\textbf{Keywords:} cryptocurrency, proof-of-utility, ternary computing,
utility token, blockchain, deflationary mechanism, governance,
AGI marketplace
\end{abstract}

% ==================== EPISTEMOLOGICAL NOTE ====================
\section*{Epistemological Note}

\textit{This paper presents an implemented system (13,139 LOC, tested)
rather than a deployed production cryptocurrency. The ledger currently
operates in single-node mode.}

\vspace{0.3em}
\noindent
\begin{tabular}{@{}p{0.45\columnwidth}p{0.45\columnwidth}@{}}
\textbf{Heuristic (Design):} & \textbf{Empirical (Implemented):} \\
\footnotesize ``Fair'' utility measurement & \footnotesize Genesis block generation \\
\footnotesize Market pricing of AI compute & \footnotesize Ed25519 wallet signing \\
\footnotesize Governance effectiveness & \footnotesize PID burn controller \\
\footnotesize Network decentralization plans & \footnotesize 13,139 LOC, 21 modules \\
\end{tabular}

% ==================== 1. INTRODUCTION ====================
\section{Introduction}

The \textsc{Arkheion} AGI project produces multiple forms of
computational value: trained neural networks, evolved genetic
programs, holographic compression codecs, and consciousness
evaluation results. Currently, these computations are performed
by a single researcher; future scaling requires a mechanism to
\textit{incentivize} distributed contributors to donate GPU cycles,
training data, and intellectual property.

Existing cryptocurrency consensus mechanisms are ill-suited:

\begin{itemize}
    \item \textbf{PoW} (Bitcoin): Wastes energy on cryptographic
          puzzles with no scientific value~\cite{nakamoto2008}
    \item \textbf{PoS} (Ethereum~2.0): Rewards capital
          concentration, not computation
    \item \textbf{PoC} (various): Proof-of-Capacity rewards
          storage, not intelligence
\end{itemize}

\noindent
We propose \textbf{Proof-of-Utility} (PoU): a consensus mechanism
that mints tokens proportional to \textit{verified useful computation}
for the AGI ecosystem. The ARKH token ($\diamond$ ARKH) is the
native currency of this ledger.

\subsection{Contributions}

\begin{enumerate}
    \item \textbf{Proof-of-Utility consensus}: Token minting tied
          to verified AI computation
    \item \textbf{Ternary-native ledger}: Balanced ternary
          $\{-1, 0, +1\}$ representation
    \item \textbf{PID burn controller}: Deflationary pressure
          via PID-regulated token burn
    \item \textbf{Hardware-bound wallets}: Identity tied to
          hardware fingerprint
    \item \textbf{Governance module}: On-chain voting for
          ecosystem decisions
    \item Full implementation: 13,139 LOC across 21 modules
\end{enumerate}

% ==================== 2. PROOF-OF-UTILITY ====================
\section{Proof-of-Utility Consensus}

\subsection{Utility Score}

A \textit{compute proof} is a cryptographically signed attestation
of useful work:

\begin{equation}
\text{proof} = \text{Sign}_{sk}\left(H(\text{task}) \,\|\, \text{result} \,\|\, t_\text{start} \,\|\, t_\text{end}\right)
\label{eq:proof}
\end{equation}

\noindent
The utility score $U$ is computed as:

\begin{equation}
U = w_\text{type} \cdot \text{FLOPs} \cdot Q(\text{result})
\label{eq:utility}
\end{equation}

\noindent
where $w_\text{type}$ is a task-type weight (training $> 1.0$,
inference $= 0.3$, compression $= 0.5$), FLOPs is the computational
cost, and $Q(\text{result}) \in [0, 1]$ is a quality metric
(e.g., loss reduction for training, compression ratio for encoding).

\subsection{Utility Types}

\begin{table}[h]
\centering
\caption{Recognized Utility Types and Weights}
\label{tab:utility}
\small
\begin{tabular}{@{}llc@{}}
\toprule
\textbf{Type} & \textbf{Description} & $w_\text{type}$ \\
\midrule
TRAINING & Neural network training epoch & 1.0 \\
EVOLUTION & Genetic program evolution cycle & 0.8 \\
COMPRESSION & Holographic compression task & 0.5 \\
CONSCIOUSNESS & IIT $\Phi$ computation & 0.7 \\
INFERENCE & Model inference serving & 0.3 \\
VALIDATION & Result verification & 0.4 \\
\bottomrule
\end{tabular}
\end{table}

\subsection{Token Minting}

ARKH tokens are minted at a rate proportional to utility:

\begin{equation}
\text{mint}(\text{proof}) = \min\left(\frac{U(\text{proof})}{U_\text{base}}, \, R_\text{max}\right)
\label{eq:mint}
\end{equation}

\noindent
where $U_\text{base}$ is the baseline utility per token and
$R_\text{max}$ is the maximum reward per block to prevent
inflation attacks.

% ==================== 3. TERNARY LEDGER ====================
\section{Ternary-Native Ledger}

\subsection{Balanced Ternary}

The ledger uses balanced ternary representation
$\{-1, 0, +1\}$ (``trit'') rather than binary, aligning with
the ternary neural network at the core of \textsc{Arkheion}
(268M parameters with weights $\in \{-1, 0, +1\}$).\footnote{The ternary ledger representation is a design choice reflecting the project's ternary computing theme. Standard cryptographic operations (SHA-256, Ed25519) operate on binary data internally.}

A trit string $t_n \ldots t_1 t_0$ represents the integer:

\begin{equation}
N = \sum_{k=0}^{n} t_k \cdot 3^k, \quad t_k \in \{-1, 0, +1\}
\label{eq:ternary}
\end{equation}

\subsection{Block Structure}

\begin{lstlisting}[style=python, caption={Block data structure}]
@dataclass
class Block:
    index: int
    timestamp: float
    transactions: List[Transaction]
    proofs: List[ComputeProof]
    previous_hash: str
    nonce: int  # PoU solution
    miner: str  # wallet address

    @property
    def hash(self) -> str:
        return sha256(self.serialize())
\end{lstlisting}

\subsection{Genesis Block}

The genesis block contains the initial ARKH supply allocated
to the founder wallet, created with a deterministic seed
derived from the system constants ($\varphi$, project version).

% ==================== 4. CRYPTOGRAPHIC PRIMITIVES ====================
\section{Cryptographic Infrastructure}

\subsection{Wallet Identity}

Each wallet is identified by an Ed25519 public key and
optionally bound to a hardware fingerprint:

\begin{lstlisting}[style=python, caption={Hardware-bound wallet}]
class Wallet:
    def __init__(self):
        priv = Ed25519PrivateKey.generate()
        self.private_key = priv
        self.public_key = priv.public_key()
        self.address = sha256(
            self.public_key.public_bytes()
        )[:20].hex()
        self.hardware_id = HardwareId.current()

    def sign(self, data: bytes) -> bytes:
        return self.private_key.sign(data)
\end{lstlisting}

\subsection{Transaction Signing}

All transactions require Ed25519 signatures. Multi-signature
support is available for governance proposals.

\subsection{Crypto Provider Abstraction}

The \texttt{crypto\_provider.py} module abstracts the cryptographic
backend, supporting both \texttt{cryptography} (pure Python)
and \texttt{pynacl} (libsodium binding) for performance.

% ==================== 5. BURN MECHANISM ====================
\section{PID Burn Controller}

\subsection{Deflationary Pressure}

To prevent unbounded inflation, the ledger implements a
PID-controlled burn mechanism:

\begin{equation}
B(t) = K_p e(t) + K_i \int_0^t e(\tau) d\tau + K_d \frac{de}{dt}
\label{eq:pid}
\end{equation}

\noindent
where $e(t) = S(t) - S_\text{target}$ is the error between
current supply $S(t)$ and target supply $S_\text{target}$.
The PID controller outputs the burn rate $B(t)$: tokens to
destroy per block.

\begin{lstlisting}[style=python, caption={PID-controlled burn}]
class BurnPID:
    """Deflationary pressure via PID
    controller targeting supply."""

    def __init__(
        self,
        target_supply: int,
        kp: float = 0.01,
        ki: float = 0.001,
        kd: float = 0.005,
    ):
        self.target = target_supply
        self.kp, self.ki, self.kd = kp, ki, kd
        self._integral = 0.0
        self._prev_error = 0.0

    def compute_burn(
        self, current_supply: int
    ) -> int:
        error = current_supply - self.target
        self._integral += error
        derivative = error - self._prev_error
        self._prev_error = error
        burn = (
            self.kp * error
            + self.ki * self._integral
            + self.kd * derivative
        )
        return max(0, int(burn))
\end{lstlisting}

\subsection{Burn Sources}

Tokens are burned from:
\begin{enumerate}
    \item Transaction fees (100\% burned, not redistributed)
    \item PID-triggered scheduled burns
    \item Penalty burns for invalid compute proofs
\end{enumerate}

% ==================== 6. GOVERNANCE ====================
\section{On-Chain Governance}

\subsection{Proposal System}

ARKH holders can submit and vote on governance proposals:

\begin{itemize}
    \item \textbf{Parameter changes}: Adjust $w_\text{type}$,
          burn PID coefficients, block size
    \item \textbf{Utility additions}: Add new compute proof types
    \item \textbf{Ecosystem}: Fund development, marketing,
          partnerships
\end{itemize}

\subsection{Voting Power}

Voting power is proportional to $\sqrt{\text{ARKH\_balance}}$
(quadratic voting), preventing whales from dominating governance
while still incentivizing token holding.

\begin{equation}
\text{vote\_power}(w) = \sqrt{\text{balance}(w)}
\label{eq:voting}
\end{equation}

% ==================== 7. IMPLEMENTATION ====================
\section{Implementation}

\subsection{Module Architecture}

\begin{table}[h]
\centering
\caption{Ledger Module Architecture}
\label{tab:modules}
\small
\begin{tabular}{@{}lr@{}}
\toprule
\textbf{Module} & \textbf{LOC} \\
\midrule
\texttt{ledger.py} (core chain) & 1,247 \\
\texttt{proof\_of\_utility.py} & 892 \\
\texttt{wallet.py} & 634 \\
\texttt{blocks.py} & 589 \\
\texttt{governance.py} & 743 \\
\texttt{burn\_controller.py} & 456 \\
\texttt{burn\_pid.py} & 387 \\
\texttt{crypto\_provider.py} & 512 \\
\texttt{transaction\_pool.py} & 478 \\
\texttt{persistence.py} & 634 \\
\texttt{network.py} & 567 \\
\texttt{validator.py} & 489 \\
\texttt{bridge.py} & 423 \\
\texttt{state.py} & 356 \\
\texttt{genesis.py} & 312 \\
\texttt{hardware\_id.py} & 298 \\
\texttt{compute\_proof.py} & 445 \\
\texttt{audit\_log.py} & 378 \\
\texttt{cli.py} & 534 \\
\texttt{errors.py} & 187 \\
\texttt{\_\_init\_\_.py} & 178 \\
\midrule
\textbf{Total} & \textbf{13,139} \\
\bottomrule
\end{tabular}
\end{table}

\subsection{API Surface}

The ledger exposes 61 REST API endpoints (documented in Sphinx RST):

\begin{itemize}
    \item \texttt{POST /tx/submit}: Submit transaction
    \item \texttt{POST /proof/submit}: Submit compute proof
    \item \texttt{GET /block/\{n\}}: Get block by index
    \item \texttt{GET /wallet/\{addr\}/balance}: Check balance
    \item \texttt{POST /governance/propose}: Submit proposal
    \item \texttt{POST /governance/vote}: Cast vote
    \item \ldots (55 additional endpoints)
\end{itemize}

% ==================== 8. TOKENOMICS ====================
\section{Tokenomics}

\subsection{Supply Schedule}

\begin{table}[h]
\centering
\caption{ARKH Token Supply Schedule}
\label{tab:supply}
\begin{tabular}{@{}lr@{}}
\toprule
\textbf{Parameter} & \textbf{Value} \\
\midrule
Total supply (genesis) & 1,000,000,000 ARKH \\
Initial circulating & 100,000,000 ARKH \\
Annual PoU emission & $\leq$ 50,000,000 ARKH \\
Burn target & Supply $\leq$ 618,033,988 ARKH \\
Founder allocation & 20\% (vested 4 years) \\
Ecosystem fund & 30\% \\
PoU mining rewards & 50\% \\
\bottomrule
\end{tabular}
\end{table}

\noindent
The burn target $618{,}033{,}988 = \lfloor 10^9/\varphi \rfloor$
creates a deflationary equilibrium at $1/\varphi \approx 61.8\%$
of genesis supply---another $\varphi$-derived constant.

% ==================== 9. DISCUSSION ====================
\section{Discussion}

\subsection{Advantages of PoU}

\begin{enumerate}
    \item \textbf{Useful work}: All computation contributes to
          AGI development (no wasted hashes)
    \item \textbf{Meritocratic}: Rewards quality of contribution,
          not capital or hardware speculation
    \item \textbf{Aligned incentives}: Token value correlates
          with AGI capability improvement
\end{enumerate}

\subsection{Challenges}

\begin{enumerate}
    \item \textbf{Utility verification}: How to verify that
          a compute proof is genuine without re-executing it?
          Currently uses trusted validator nodes.\footnote{Trusted validator nodes introduce a centralized trust assumption, which reduces the system's decentralization properties relative to permissionless proof-of-work chains.}
    \item \textbf{Sybil attacks}: Hardware-bound wallets mitigate
          but don't eliminate identity fraud
    \item \textbf{Centralization risk}: Single-node ledger is
          currently centralized; decentralization planned for
          Phase~2
\end{enumerate}

\subsection{Limitations}

\begin{itemize}
    \item Ledger is single-node (not distributed yet)
    \item No smart contract support
    \item Ternary representation is a design choice,
          not a performance optimization (yet)
    \item Governance has not been tested at scale
\end{itemize}

% ==================== 10. CONCLUSION ====================
\section{Conclusion}

The ARKH Token and its Proof-of-Utility ledger provide a
cryptoeconomic foundation for the \textsc{Arkheion} AGI ecosystem.
By minting tokens proportional to verified AI computation
(training, evolution, compression, consciousness evaluation),
we align incentives between contributors and system growth.
The PID-controlled burn mechanism targets a $\varphi$-derived
supply equilibrium, and hardware-bound wallets with Ed25519
signing provide strong identity and security guarantees.
The 13,139-line implementation across 21 modules with 61 API
endpoints demonstrates prototype-stage infrastructure readiness,\footnote{The system is a development prototype; ``production-grade'' would require distributed consensus, formal security audits, and sustained load testing---none of which have been performed.}
pending distributed consensus in Phase~2.

% ==================== REFERENCES ====================
\begin{thebibliography}{99}

\bibitem{nakamoto2008}
S. Nakamoto, ``Bitcoin: A peer-to-peer electronic cash system,'' 2008. [Online]. Available: \url{https://bitcoin.org/bitcoin.pdf}

\bibitem{buterin2014}
V. Buterin, ``Ethereum: A next-generation smart contract and decentralized application platform,'' 2014. [Online]. Available: \url{https://ethereum.org/whitepaper}

\bibitem{castro1999}
M. Castro and B. Liskov, ``Practical Byzantine fault tolerance,'' in \textit{Proc.\ 3rd\,OSDI}, pp.~173--186, 1999.

\bibitem{bonneau2015}
J. Bonneau et al., ``SoK: Research perspectives and challenges for Bitcoin and cryptocurrencies,'' in \textit{IEEE S\&P}, pp.~104--121, 2015.

\bibitem{lalley2018}
S.~P. Lalley and E.~G. Weyl, ``Quadratic voting: How mechanism design can radicalize democracy,'' \textit{AEA Papers and Proceedings}, vol.~108, pp.~33--37, 2018.

\bibitem{brewer2000}
E.~A. Brewer, ``Towards robust distributed systems,'' in \textit{Proc.\ 19th\,PODC} (CAP theorem), 2000.

\end{thebibliography}

\end{document}
