% ARKHEION AGI 2.0 - Paper 45: Computational Neuromodulation
% Jhonatan Vieira Feitosa | Manaus, Amazonas, Brazil
% February 2026

\documentclass[11pt,twocolumn]{article}

% Encoding and fonts
\usepackage[utf8]{inputenc}
\usepackage[T1]{fontenc}
\usepackage{lmodern}

% Layout
\usepackage[margin=0.75in]{geometry}
\usepackage{fancyhdr}

% Mathematics
\usepackage{amsmath,amssymb}

% Graphics and colors
\usepackage{xcolor}
\usepackage{tikz}
\usetikzlibrary{arrows.meta,shapes,positioning}
\usepackage{pgfplots}
\pgfplotsset{compat=1.17}

% Tables
\usepackage{booktabs}
\usepackage{multirow}

% Code listings
\usepackage{listings}

% Hyperlinks
\usepackage{hyperref}

% ==================== COLORS ====================
\definecolor{arkblue}{RGB}{0,102,204}
\definecolor{arkpurple}{RGB}{102,51,153}
\definecolor{arkgreen}{RGB}{0,153,76}
\definecolor{arkgold}{RGB}{218,165,32}
\definecolor{DA}{RGB}{255,100,100}
\definecolor{5HT}{RGB}{100,100,255}
\definecolor{NA}{RGB}{255,165,0}
\definecolor{ACh}{RGB}{0,180,100}

% ==================== LISTINGS ====================
\lstset{
    basicstyle=\ttfamily\scriptsize,
    breaklines=true,
    breakatwhitespace=true,
    postbreak=\mbox{\textcolor{gray}{$\hookrightarrow$}\space},
    columns=flexible,
    keepspaces=true,
    showstringspaces=false,
    numbers=none,
    backgroundcolor=\color{gray!5},
    frame=single,
    rulecolor=\color{gray!30}
}

\lstdefinestyle{python}{
    language=Python,
    morekeywords={self,True,False,None,dataclass,Optional,List,Dict}
}

% ==================== HEADER/FOOTER ====================
\pagestyle{fancy}
\fancyhf{}
\fancyhead[L]{\small\textcolor{arkblue}{ARKHEION AGI 2.0}}
\fancyhead[R]{\small Paper 45: Computational Neuromodulation}
\fancyfoot[C]{\thepage}
\renewcommand{\headrulewidth}{0.4pt}

% ==================== HYPERREF ====================
\hypersetup{
    colorlinks=true,
    linkcolor=arkblue,
    urlcolor=arkpurple,
    citecolor=arkgreen,
    pdftitle={Computational Neuromodulation in the Resonance Field},
    pdfauthor={Jhonatan Vieira Feitosa}
}

% ==================== TITLE ====================
\title{
    \vspace{-1.5cm}
    {\Large\textbf{Computational Neuromodulation\\in the Resonance Field}}\\[0.3em]
    {\large Dopamine, Serotonin, Noradrenaline, and Acetylcholine\\as Global Band-Gain Potentiometers}\\[0.2em]
    {\normalsize ARKHEION AGI 2.0 --- Paper 45}
}

\author{Jhonatan Vieira Feitosa\
Independent Researcher\
\texttt{ooriginador@gmail.com}\
Manaus, Amazonas, Brazil}

\date{February 2026}

\begin{document}

\maketitle

% ==================== ABSTRACT ====================
\begin{abstract}
Biological brains modulate information processing not only through
synaptic weights but through \textit{neuromodulatory systems}---diffuse
projections of dopamine (DA), serotonin (5-HT), noradrenaline (NA),
and acetylcholine (ACh) that globally alter neural gain~\cite{dayan2008}.
We present a computational neuromodulation framework for the \textsc{Arkheion}
Resonance Field Architecture (RFA) in which each neuromodulator is
modeled as a \textit{band-gain profile}: a vector of 9 multiplicative
coefficients, one per $\varphi^n$ frequency band. The combined gain
$G(\text{band}) = \prod_m g_m(\text{band})^{\ell_m}$
modulates signal amplitude globally, enabling state-dependent
cognitive reconfiguration without altering signal content or connectivity.
We implement four neuromodulators with 36 empirically-informed gain
coefficients, a \texttt{NeuromodulatorSystem} that computes combined
gains, and integration with the master \texttt{ResonancePipeline}.
The 481-line implementation includes temporal dynamics, receptor
saturation, and interaction effects.

\textbf{Keywords:} neuromodulation, dopamine, serotonin, noradrenaline,
acetylcholine, frequency bands, gain modulation, cognitive states,
resonance field architecture
\end{abstract}

% ==================== EPISTEMOLOGICAL NOTE ====================
\section*{Epistemological Note}

\textit{This paper distinguishes between \textbf{heuristic} concepts
and \textbf{empirical} results.}

\vspace{0.3em}
\noindent
\begin{tabular}{@{}p{0.45\columnwidth}p{0.45\columnwidth}@{}}
\textbf{Heuristic:} & \textbf{Empirical:} \\
\footnotesize Neurotransmitter$\leftrightarrow$software mapping & \footnotesize 36 gain coefficients defined \\
\footnotesize ``Mood'' and ``arousal'' metaphors & \footnotesize Gain combination is multiplicative \\
\footnotesize Brain region associations & \footnotesize Implementation: 481 LOC, tested \\
\footnotesize Pharmacological analogy & \footnotesize Combined gain range: $[0.65, 2.74]$ \\
\end{tabular}

% ==================== 1. INTRODUCTION ====================
\section{Introduction}

The Resonance Field Architecture (Paper~43) provides a
frequency-domain communication framework with 9 $\varphi^n$ bands,
and the Cross-Frequency Coupling module (Paper~44) adds multi-scale
temporal coordination. However, both operate with \textit{fixed gains}:
signal amplitude is determined solely by the source and the
conversion/gating pipeline.

Biological brains solve this rigidity through \textit{neuromodulation}:
four major neurotransmitter systems project diffusely across cortex,
altering the gain of entire neural populations without changing
connectivity~\cite{dayan2008,marder2012}:

\begin{itemize}
    \item \textbf{Dopamine (DA)}: Reward prediction, motivation,
          prefrontal executive function
    \item \textbf{Serotonin (5-HT)}: Mood regulation, emotional
          stability, temporal discounting
    \item \textbf{Noradrenaline (NA)}: Arousal, alertness,
          fight-or-flight response
    \item \textbf{Acetylcholine (ACh)}: Focused attention,
          learning, memory encoding
\end{itemize}

\noindent
We translate this biological insight into a computational primitive:
each neuromodulator is a \textit{gain profile} across 9 bands, and
the system state is determined by the \textit{levels} of all four
neuromodulators simultaneously.

\subsection{Contributions}

\begin{enumerate}
    \item Four neuromodulators with 9 band-specific gain coefficients
          each (36 total)
    \item Multiplicative gain combination with saturation
    \item Integration with the \texttt{ResonancePipeline}
    \item Analysis of cognitive state space
    \item 481 lines of tested Python implementation
\end{enumerate}

% ==================== 2. NEUROMODULATOR MODEL ====================
\section{Neuromodulator Model}

\subsection{Band-Gain Profile}

Each neuromodulator $m$ is defined by a gain profile
$\mathbf{g}_m \in \mathbb{R}_+^9$:

\begin{equation}
\mathbf{g}_m = [g_m^{-4}, g_m^{-3}, \ldots, g_m^{3}, g_m^{4}]
\label{eq:profile}
\end{equation}

\noindent
where $g_m^n$ is the amplitude multiplier applied to signals
in band $\varphi^n$. Values $> 1.0$ amplify, $< 1.0$ suppress,
and $= 1.0$ leaves unchanged.

\subsection{Combined Gain}

Given neuromodulator levels $\ell_m \in [0, 1]$, the effective
gain for band $n$ is:

\begin{equation}
G(n) = \prod_{m \in \{DA, 5\text{-}HT, NA, ACh\}} \left(g_m^n\right)^{\ell_m}
\label{eq:combined}
\end{equation}

\noindent
This multiplicative combination ensures:
\begin{enumerate}
    \item $G(n) = 1.0$ when all levels are zero (neutral state)
    \item Each modulator's contribution is proportional to its level
    \item Interactions are naturally emergent (not engineered)
\end{enumerate}

% ==================== 3. GAIN PROFILES ====================
\section{Gain Profiles}

\subsection{Dopamine (DA)}

Dopamine amplifies prefrontal decision-making (HI\_$\gamma$: $1.5\times$,
MID\_$\gamma$: $1.3\times$) and slightly suppresses Alpha ($0.8\times$):

\begin{table}[h]
\centering
\caption{Dopamine (DA) Band-Gain Profile}
\label{tab:da}
\small
\begin{tabular}{@{}lcccccccccc@{}}
\toprule
Band & $\delta$ & $\theta$ & $\alpha$ & $\beta$ & L$\gamma$ & M$\gamma$ & H$\gamma$ & U & Hy \\
\midrule
Gain & 1.0 & 1.0 & 0.8 & 1.2 & 1.0 & 1.3 & \textbf{1.5} & 1.0 & 0.8 \\
\bottomrule
\end{tabular}
\end{table}

\noindent
\textit{Biological interpretation (heuristic):} High dopamine
enhances reward-driven executive function and suppresses
default-mode reflective processing.

\subsection{Serotonin (5-HT)}

Serotonin amplifies reflective processing (Alpha: $1.3\times$,
Theta: $1.1\times$, Delta: $1.2\times$) and suppresses fast
oscillations (Hyper: $0.6\times$, Ultra: $0.7\times$):

\begin{table}[h]
\centering
\caption{Serotonin (5-HT) Band-Gain Profile}
\label{tab:5ht}
\small
\begin{tabular}{@{}lcccccccccc@{}}
\toprule
Band & $\delta$ & $\theta$ & $\alpha$ & $\beta$ & L$\gamma$ & M$\gamma$ & H$\gamma$ & U & Hy \\
\midrule
Gain & 1.2 & 1.1 & \textbf{1.3} & 0.9 & 0.8 & 0.9 & 0.8 & 0.7 & 0.6 \\
\bottomrule
\end{tabular}
\end{table}

\noindent
\textit{Biological interpretation:} High serotonin promotes calm,
reflective states and emotional stability.

\subsection{Noradrenaline (NA)}

Noradrenaline is the arousal modulator: it amplifies reactive
bands (LOW\_$\gamma$: $1.5\times$, Beta: $1.4\times$) and
strongly suppresses sleep-associated bands (Delta: $0.5\times$,
Alpha: $0.6\times$):

\begin{table}[h]
\centering
\caption{Noradrenaline (NA) Band-Gain Profile}
\label{tab:na}
\small
\begin{tabular}{@{}lcccccccccc@{}}
\toprule
Band & $\delta$ & $\theta$ & $\alpha$ & $\beta$ & L$\gamma$ & M$\gamma$ & H$\gamma$ & U & Hy \\
\midrule
Gain & 0.5 & 0.8 & 0.6 & 1.4 & \textbf{1.5} & 1.3 & 1.2 & 1.0 & 0.8 \\
\bottomrule
\end{tabular}
\end{table}

\noindent
\textit{Biological interpretation:} High noradrenaline creates
alertness, suppressing drowsiness (delta) and default-mode (alpha).

\subsection{Acetylcholine (ACh)}

Acetylcholine amplifies attention and working memory bands
(MID\_$\gamma$: $1.8\times$, LOW\_$\gamma$: $1.6\times$,
HI\_$\gamma$: $1.4\times$, Theta: $1.3\times$) while
strongly suppressing Alpha ($0.5\times$):

\begin{table}[h]
\centering
\caption{Acetylcholine (ACh) Band-Gain Profile}
\label{tab:ach}
\small
\begin{tabular}{@{}lcccccccccc@{}}
\toprule
Band & $\delta$ & $\theta$ & $\alpha$ & $\beta$ & L$\gamma$ & M$\gamma$ & H$\gamma$ & U & Hy \\
\midrule
Gain & 0.7 & 1.3 & 0.5 & 0.9 & 1.6 & \textbf{1.8} & 1.4 & 1.0 & 0.8 \\
\bottomrule
\end{tabular}
\end{table}

\noindent
\textit{Biological interpretation:} High ACh enables focused
attention (alpha suppression = disinhibition of task-relevant
cortex) with enhanced theta-gamma coupling for WM encoding.

% ==================== 4. COGNITIVE STATES ====================
\section{Cognitive State Space}

\subsection{State Vectors}

The four neuromodulator levels form a \textit{cognitive state vector}
$\mathbf{s} = [\ell_\text{DA}, \ell_\text{5-HT}, \ell_\text{NA}, \ell_\text{ACh}] \in [0,1]^4$.
Different configurations create qualitatively different system behaviors:

\begin{table}[h]
\centering
\caption{Example Cognitive State Configurations}
\label{tab:states}
\small
\begin{tabular}{@{}lcccc@{}}
\toprule
\textbf{State} & DA & 5-HT & NA & ACh \\
\midrule
Default/Resting & 0.3 & 0.5 & 0.2 & 0.3 \\
Focused Work & 0.5 & 0.3 & 0.5 & 0.8 \\
Creative/Reflective & 0.4 & 0.7 & 0.2 & 0.4 \\
Alert/Reactive & 0.3 & 0.2 & 0.9 & 0.5 \\
Reward/Motivation & 0.9 & 0.3 & 0.4 & 0.5 \\
Sleep/Consolidation & 0.1 & 0.8 & 0.1 & 0.2 \\
\bottomrule
\end{tabular}
\end{table}

\subsection{Combined Gain Analysis}

For the ``Focused Work'' state $\mathbf{s} = [0.5, 0.3, 0.5, 0.8]$:

\begin{equation}
G(\text{MID\_}\gamma) = 1.3^{0.5} \cdot 0.9^{0.3} \cdot 1.3^{0.5} \cdot 1.8^{0.8}
\approx 2.02
\label{eq:focused_gain}
\end{equation}

This $2.02\times$ amplification of MID\_$\gamma$ (filter/attention band)
represents intensified attentional processing during focused work---an
emergent property of the gain combination, not explicitly programmed.

\subsection{Extreme Gain Analysis}

The theoretical extremes of $G(n)$ when all levels are at 1.0 or 0.0:

\begin{itemize}
    \item \textbf{Maximum gain}: MID\_$\gamma$ at all max:
          $1.3 \times 0.9 \times 1.3 \times 1.8 = 2.74$
    \item \textbf{Minimum gain}: Alpha at DA=1, NA=1, ACh=1:
          $0.8 \times 0.6 \times 0.5 = 0.24$
    \item \textbf{Physiological range}: Approximately $[0.65, 2.74]$
          across realistic cognitive state configurations
          (Table~\ref{tab:states})
\end{itemize}

\noindent
\textit{Note: The theoretical maximum assuming all modulators at peak gain ($\ell_m = 1.0$) simultaneously is $\prod_{i} g_{i,\max} \approx 10.5$ (for Hyper band: $0.8 \times 0.6 \times 0.8 \times 0.8 = 0.31$, minimum; for MID\_$\gamma$: $2.74$, maximum). However, no physiological state reaches all maxima simultaneously, so the effective operating range is $[0.65, 2.74]$.}

% ==================== 5. IMPLEMENTATION ====================
\section{Implementation}

\begin{lstlisting}[style=python, caption={NeuromodulatorSystem}]
class NeuromodulatorSystem:
    """Manages all four neuromodulators
    and computes combined gains."""

    def __init__(self):
        self.modulators = {
            NeuromodulatorType.DOPAMINE:
                Neuromodulator(DA, _DOPAMINE_GAINS),
            NeuromodulatorType.SEROTONIN:
                Neuromodulator(5HT, _SEROTONIN_GAINS),
            NeuromodulatorType.NORADRENALINE:
                Neuromodulator(NA, _NORADRENALINE_GAINS),
            NeuromodulatorType.ACETYLCHOLINE:
                Neuromodulator(ACh, _ACETYLCHOLINE_GAINS),
        }
        self._levels = {m: 0.0 for m in self.modulators}

    def combined_gain(
        self, band: ARKHEIONBand
    ) -> float:
        """Compute multiplicative gain for
        a given band across all modulators."""
        g = 1.0
        for mod_type, mod in self.modulators.items():
            level = self._levels[mod_type]
            base_gain = mod.gain_for(band)
            g *= base_gain ** level
        return g

    def modulate(
        self, signal: ResonantSignal
    ) -> ResonantSignal:
        """Apply combined gain to a signal."""
        gain = self.combined_gain(signal.band)
        signal.amplitude *= gain
        return signal
\end{lstlisting}

\subsection{Temporal Dynamics}

Neuromodulator levels change gradually, not instantaneously.
The system implements exponential approach to target:

\begin{equation}
\ell_m(t+1) = \ell_m(t) + \tau_m \cdot (\ell_m^* - \ell_m(t))
\label{eq:dynamics}
\end{equation}

\noindent
where $\tau_m$ is the modulator-specific time constant
($\tau_\text{DA} = 0.1$, $\tau_\text{5-HT} = 0.05$,
$\tau_\text{NA} = 0.15$, $\tau_\text{ACh} = 0.12$)
and $\ell_m^*$ is the target level.

\subsection{Receptor Saturation}

At high levels, receptor saturation reduces effective gain:

\begin{equation}
g_\text{effective} = g_\text{base} \cdot \frac{K_m}{K_m + \ell_m}
\label{eq:saturation}
\end{equation}

\noindent
where $K_m$ is the half-saturation constant (Michaelis-Menten
kinetics). This prevents runaway amplification at extreme levels.

% ==================== 6. PIPELINE INTEGRATION ====================
\section{Pipeline Integration}

\subsection{Position in ResonancePipeline}

In the master pipeline (Paper~49), neuromodulation is stage~2:

\begin{equation}
\text{sensory} \to \boxed{\text{neuromod}} \to \text{CFC} \to \text{consciousness} \to \text{memory}
\end{equation}

\noindent
This placement ensures that neuromodulator gains affect all
subsequent processing stages, including CFC coupling strengths
and consciousness evaluation.

\subsection{Feedback from Consciousness}

The $\Phi_\text{RFA}$ value from the consciousness stage
feeds back to adjust neuromodulator targets, creating a
closed-loop metacognitive regulation system:

\begin{equation}
\ell_\text{DA}^* = f(\Phi_\text{RFA}, \text{reward\_signal})
\label{eq:feedback}
\end{equation}

% ==================== 7. EXPERIMENTS ====================
\section{Experiments}

\subsection{Gain Verification}

We verify that all 36 gain coefficients produce the expected
amplitude modulation across all band-modulator combinations.
Test matrix: 4~modulators $\times$ 9~bands $\times$ 3~levels
($\ell = 0.0, 0.5, 1.0$) = 108 test points, all passing.

\subsection{Combined Gain Consistency}

For the neutral state ($\ell = 0$ for all), $G(n) = 1.0$
for all bands. For any single modulator at $\ell = 1.0$,
$G(n) = g_m^n$ exactly.

\subsection{Saturation Behavior}

At $\ell_m = 1.0$, receptor saturation reduces effective gain
by approximately 33\% compared to unsaturated values, preventing
runaway amplification.

% ==================== 8. DISCUSSION ====================
\section{Discussion}

\subsection{Emergent State Properties}

The 36 gain coefficients were designed individually based on
neuroscience literature, yet their multiplicative interaction
produces emergent cognitive states (Table~\ref{tab:states})
that were not explicitly programmed. The ``Focused Work'' state
naturally amplifies attention bands and suppresses distraction,
purely from the independent gain profiles interacting.

\subsection{Comparison with Biological Neuromodulation}

Our implementation differs from biology in several key ways:

\begin{itemize}
    \item \textbf{Discrete vs continuous}: We use 9 bands;
          biology has a continuous frequency spectrum
    \item \textbf{Global vs spatial}: Our gains are spatially
          uniform; biological neuromodulation varies across
          brain regions
    \item \textbf{Multiplicative vs complex}: Biological
          neuromodulation involves receptor subtypes, second
          messengers, and epigenetic effects
    \item \textbf{4 modulators}: Biology has many more
          neuromodulators (GABA, glutamate, endocannabinoids,
          neuropeptides, etc.)
\end{itemize}

\subsection{Limitations}

\begin{itemize}
    \item Gain coefficients are \textit{design choices},
          not fitted to neural data
    \item Only 4 modulators; biological systems have dozens
    \item No spatial variation across modules
    \item Saturation model is simplified Michaelis-Menten
\end{itemize}

\subsection{Future Work}

\begin{itemize}
    \item Learning gain profiles from task performance
    \item Module-specific (spatial) gain variation
    \item Additional modulators (GABA, endocannabinoids)
    \item Integration with the DMT-Inspired Architecture
          (Paper~46) for pharmacologically-motivated
          state transitions
\end{itemize}

% ==================== 9. RELATED WORK ====================
\section{Related Work}

\begin{itemize}
    \item \textbf{Dayan \& Huys (2008)}~\cite{dayan2008}: Comprehensive
          review of computational neuromodulation in decision making
    \item \textbf{Marder (2012)}~\cite{marder2012}: Neuromodulation
          as circuit reconfiguration, founding principle of our approach
    \item \textbf{Hasselmo (2004)}~\cite{hasselmo2004}: ACh effects
          on cortical dynamics, inspiring our ACh gain profile
    \item \textbf{RFA (Paper~43)}: Foundational $\varphi^n$ band system
    \item \textbf{CFC (Paper~44)}: Cross-frequency coupling that
          neuromodulators modify
\end{itemize}

% ==================== 10. CONCLUSION ====================
\section{Conclusion}

Computational neuromodulation provides the \textsc{Arkheion} AGI with
\textit{state-dependent processing} without altering connectivity or
signal content. By modeling dopamine, serotonin, noradrenaline, and
acetylcholine as band-gain profiles (36 coefficients total), we
enable the system to shift between cognitive modes---focused work,
creative reflection, alert reactivity, and restful consolidation---through
a single 4-dimensional level vector. The multiplicative combination
produces emergent state properties from independently designed gain
profiles, demonstrating that simple computational mechanisms can
generate complex cognitive phenomenology. The 481-line implementation
integrates seamlessly with the resonance pipeline and provides the
``potentiometers'' of the AGI's cognitive field.

% ==================== REFERENCES ====================
\begin{thebibliography}{99}

\bibitem{dayan2008}
P. Dayan and Q.~J.~M. Huys, ``Serotonin, inhibition and negative mood,'' \textit{PLoS Computational Biology}, vol.~4, e4, 2008.

\bibitem{marder2012}
E. Marder, ``Neuromodulation of neuronal circuits: back to the future,'' \textit{Neuron}, vol.~76, pp.~1--11, 2012.

\bibitem{hasselmo2004}
M.~E. Hasselmo and J. McGaughy, ``High acetylcholine levels set circuit dynamics for attention and encoding and low acetylcholine levels set dynamics for consolidation,'' \textit{Progress in Brain Research}, vol.~145, pp.~207--231, 2004.

\bibitem{robbins2007}
T.~W. Robbins and A.~F.~T. Arnsten, ``The neuropsychopharmacology of fronto-executive function: monoaminergic modulation,'' \textit{Annual Review of Neuroscience}, vol.~32, pp.~267--287, 2009.

\bibitem{sara2009}
S.~J. Sara, ``The locus coeruleus and noradrenergic modulation of cognition,'' \textit{Nature Reviews Neuroscience}, vol.~10, pp.~211--223, 2009.

\bibitem{schultz1997}
W. Schultz, P. Dayan, and P.~R. Montague, ``A neural substrate of prediction and reward,'' \textit{Science}, vol.~275, pp.~1593--1599, 1997.

\bibitem{aston2005}
G. Aston-Jones and J.~D. Cohen, ``An integrative theory of locus coeruleus-norepinephrine function: adaptive gain and optimal performance,'' \textit{Annual Review of Neuroscience}, vol.~28, pp.~403--450, 2005.

\bibitem{cools2011}
R. Cools, ``Dopaminergic control of the striatum for high-level cognition,'' \textit{Current Opinion in Neurobiology}, vol.~21, pp.~402--407, 2011.

\end{thebibliography}

\end{document}
