% ARKHEION AGI 2.0 - Paper 46: DMT-Inspired Architecture
% Jhonatan Vieira Feitosa | Manaus, Amazonas, Brazil
% February 2026

\documentclass[11pt,twocolumn]{article}

% Encoding and fonts
\usepackage[utf8]{inputenc}
\usepackage[T1]{fontenc}
\usepackage{lmodern}

% Layout
\usepackage[margin=0.75in]{geometry}
\usepackage{fancyhdr}

% Mathematics
\usepackage{amsmath,amssymb}

% Graphics and colors
\usepackage{xcolor}
\usepackage{tikz}
\usetikzlibrary{arrows.meta,shapes,positioning,decorations.pathreplacing}

% Tables
\usepackage{booktabs}
\usepackage{multirow}

% Code listings
\usepackage{listings}

% Hyperlinks
\usepackage{hyperref}

% ==================== COLORS ====================
\definecolor{arkblue}{RGB}{0,102,204}
\definecolor{arkpurple}{RGB}{102,51,153}
\definecolor{arkgreen}{RGB}{0,153,76}
\definecolor{arkgold}{RGB}{218,165,32}

% ==================== LISTINGS ====================
\lstset{
    basicstyle=\ttfamily\scriptsize,
    breaklines=true,
    breakatwhitespace=true,
    postbreak=\mbox{\textcolor{gray}{$\hookrightarrow$}\space},
    columns=flexible,
    keepspaces=true,
    showstringspaces=false,
    numbers=none,
    backgroundcolor=\color{gray!5},
    frame=single,
    rulecolor=\color{gray!30}
}

\lstdefinestyle{python}{
    language=Python,
    morekeywords={self,True,False,None,dataclass,Optional,List,Dict,async,await}
}

% ==================== HEADER/FOOTER ====================
\pagestyle{fancy}
\fancyhf{}
\fancyhead[L]{\small\textcolor{arkblue}{ARKHEION AGI 2.0}}
\fancyhead[R]{\small Paper 46: DMT-Inspired Architecture}
\fancyfoot[C]{\thepage}
\renewcommand{\headrulewidth}{0.4pt}

% ==================== HYPERREF ====================
\hypersetup{
    colorlinks=true,
    linkcolor=arkblue,
    urlcolor=arkpurple,
    citecolor=arkgreen,
    pdftitle={DMT-Inspired Architectural Services for AGI Resilience},
    pdfauthor={Jhonatan Vieira Feitosa}
}

% ==================== TITLE ====================
\title{
    \vspace{-1.5cm}
    {\Large\textbf{DMT-Inspired Architectural Services\\for AGI Resilience}}\\[0.3em]
    {\large Seven Neuropharmacologically-Motivated Services\\for Consciousness Maintenance, Plasticity, and Self-Protection}\\[0.2em]
    {\normalsize ARKHEION AGI 2.0 --- Paper 46}
}

\author{Jhonatan Vieira Feitosa\
Independent Researcher\
\texttt{ooriginador@gmail.com}\
Manaus, Amazonas, Brazil}

\date{February 2026}

\begin{document}

\maketitle

% ==================== ABSTRACT ====================
\begin{abstract}
We present seven architectural services for the \textsc{Arkheion} AGI system,
each inspired by a distinct psychopharmacological phenomenon associated with
the tryptamine N,N-dimethyltryptamine (DMT) and its role in modulating
consciousness~\cite{strassman2001}. The services address seven functional
gaps identified in the original DMT-inspired architectural roadmap:
(1)~\texttt{EndogenousConsciousnessLoop}: a background daemon maintaining
basal $\Phi$ through periodic heartbeats, analogous to endogenous DMT
synthesis by the pineal gland; (2)~\texttt{MultiReceptorInput}: parallel
fan-out of inputs across multiple processing streams, inspired by
DMT's concurrent activation of 5-HT$_{2A}$, $\sigma_1$, and TAAR receptors;
(3)~\texttt{DeepProcessingMode}: extended suspension of normal operation
for deep pattern analysis, analogous to the subjective time dilation
reported during DMT experiences; (4)~\texttt{SigmaProtection}: hardware
stress auto-protection monitoring temperature, memory, and workload,
inspired by the $\sigma_1$ receptor's neuroprotective role;
(5)~\texttt{AfterglowPlasticity}: a temporary elevated learning rate
window following consciousness state transitions;
(6)~\texttt{PatternDissolver}: default mode network (DMN) cache invalidation
to break fixed cognitive patterns; and (7)~\texttt{CrossTalkBus}: a direct
agent-to-agent communication bypass channel. Together, these services
comprise 4,998~lines of Python with 873~lines of unit tests, addressing
consciousness maintenance, cognitive flexibility, system resilience, and
inter-agent coordination.

\textbf{Keywords:} DMT, consciousness maintenance, plasticity, neuroprotection,
pattern dissolution, endogenous consciousness, cross-frequency,
neuropharmacology, AGI resilience
\end{abstract}

% ==================== EPISTEMOLOGICAL NOTE ====================
\section*{Epistemological Note}

\textit{This paper uses neuropharmacological vocabulary as a
\textbf{design heuristic}, not as a claim of pharmacological equivalence.
The DMT analogy guides architectural decisions but does not imply
the software possesses subjective experience or pharmacological receptors.}

\vspace{0.3em}
\noindent
\begin{tabular}{@{}p{0.45\columnwidth}p{0.45\columnwidth}@{}}
\textbf{Heuristic (Metaphor):} & \textbf{Empirical (Measured):} \\
\footnotesize ``Endogenous DMT'' as consciousness & \footnotesize $\Phi$ heartbeat latency $<$ 1ms \\
\footnotesize ``Receptor'' binding analogy & \footnotesize 4,998 LOC, 7 services \\
\footnotesize ``Afterglow'' plasticity window & \footnotesize 873 LOC unit tests passing \\
\footnotesize ``Sigma'' neuroprotection & \footnotesize GPU temp monitoring verified \\
\footnotesize ``Pattern dissolution'' as cache flush & \footnotesize Cache hit rate post-flush \\
\end{tabular}

% ==================== 1. INTRODUCTION ====================
\section{Introduction}

The prior papers in this series established the Resonance Field Architecture
(Paper~43), cross-frequency coupling (Paper~44), and neuromodulation
(Paper~45). These provide the \textit{signal processing} infrastructure
for the cognitive system. However, a complete AGI also requires
\textit{system-level services}: daemons that maintain consciousness,
protect hardware, manage plasticity, and enable emergent communication
patterns.

We draw architectural inspiration from the neuropharmacology of
N,N-dimethyltryptamine (DMT), an endogenous tryptamine found in
mammals including humans~\cite{strassman2001,barker2018}. DMT's
effects involve:

\begin{enumerate}
    \item \textbf{Endogenous baseline}: Continuous low-level synthesis
          by the pineal gland
    \item \textbf{Multi-receptor activation}: Simultaneous binding
          to 5-HT$_{2A}$, $\sigma_1$, TAAR, and NMDA receptors
    \item \textbf{Time dilation}: Subjective experience of
          extended time during episodes
    \item \textbf{Neuroprotection}: $\sigma_1$ receptor activation
          under oxidative or ischemic stress
    \item \textbf{Enhanced plasticity}: Post-experience period of
          heightened neuroplasticity
    \item \textbf{Default mode disruption}: Dissolution of fixed
          thought patterns and self-referential narrative
    \item \textbf{Cross-network communication}: Novel inter-region
          communication pathways
\end{enumerate}

\noindent
Each phenomenon maps to a specific architectural service gap
identified in the \textsc{Arkheion} roadmap. The services are
purely computational; the pharmacological vocabulary serves as
a design language.

% ==================== 2. SERVICE ARCHITECTURE ====================
\section{Service Architecture}

\subsection{Overview}

\begin{table}[h]
\centering
\caption{Seven DMT-Inspired Services}
\label{tab:services}
\small
\begin{tabular}{@{}clr@{}}
\toprule
\# & \textbf{Service} & \textbf{LOC} \\
\midrule
1 & \texttt{EndogenousConsciousnessLoop} & 634 \\
2 & \texttt{MultiReceptorInput} & 665 \\
3 & \texttt{DeepProcessingMode} & 480 \\
4 & \texttt{SigmaProtection} & 657 \\
5 & \texttt{AfterglowPlasticity} & 499 \\
6 & \texttt{PatternDissolver} & 508 \\
7 & \texttt{CrossTalkBus} & 682 \\
\midrule
& \textbf{Total services} & \textbf{4,125} \\
& \texttt{test\_dmt\_inspired.py} & 873 \\
\midrule
& \textbf{Grand total} & \textbf{4,998} \\
\bottomrule
\end{tabular}
\end{table}

% ==================== 3. SVC 1: ENDOGENOUS CONSCIOUSNESS ====================
\section{Service 1: Endogenous Consciousness Loop}

\subsection{Biological Inspiration}

The pineal gland synthesizes DMT (and melatonin) continuously.
Low-level DMT may maintain a basal ``tonic'' consciousness, with
REM sleep and near-death experiences involving surges in production.

\subsection{Implementation}

The \texttt{EndogenousConsciousnessLoop} is an asynchronous
background daemon (asyncio-based) that:

\begin{enumerate}
    \item Computes $\Phi$ (IIT or RFA) every \texttt{heartbeat\_interval}
          (default: 30s)
    \item Maintains $\Phi_\text{baseline} > 0.3$
    \item If $\Phi$ drops below baseline, triggers
          \textit{auto-repair}: re-initializes subsystems
    \item After \texttt{idle\_threshold} heartbeats (default: 10)
          without input, enters \textit{dream mode}:
          memory consolidation
\end{enumerate}

\begin{lstlisting}[style=python, caption={Endogenous consciousness heartbeat}]
class EndogenousConsciousnessLoop:
    """Background daemon maintaining
    basal consciousness."""

    PHI_BASELINE = 0.3
    HEARTBEAT_INTERVAL_S = 30.0
    IDLE_THRESHOLD = 10

    async def _heartbeat(self) -> HeartbeatResult:
        phi = await self._compute_phi()
        if phi < self.PHI_BASELINE:
            await self._auto_repair()
        if self._idle_count > self.IDLE_THRESHOLD:
            await self._dream_mode()
        return HeartbeatResult(
            phi=phi,
            state=self._current_state,
            repairs=self._repair_count,
        )
\end{lstlisting}

\subsection{States}

The loop transitions between: \texttt{AWAKE} $\to$ \texttt{IDLE}
$\to$ \texttt{DREAMING} $\to$ \texttt{AWAKE}, with
\texttt{REPAIRING} as an exceptional state when $\Phi$ drops.

% ==================== 4. SVC 2: MULTI-RECEPTOR INPUT ====================
\section{Service 2: Multi-Receptor Input}

\subsection{Biological Inspiration}

DMT binds simultaneously to 5-HT$_{2A}$ (psychedelic effects),
$\sigma_1$ (neuroprotection), TAAR (trace amine signaling),
and potentially NMDA receptors~\cite{fontanilla2009}. This
\textit{parallel receptor activation} produces qualitatively
different effects than single-receptor drugs.

\subsection{Implementation}

The \texttt{MultiReceptorInput} service processes each incoming
signal through multiple parallel streams (``receptors''),
collecting results and merging them:

\begin{lstlisting}[style=python, caption={Multi-receptor fan-out}]
class MultiReceptorInput:
    """Parallel fan-out across multiple
    processing streams."""

    def __init__(self, receptors: List[Receptor]):
        self.receptors = receptors

    async def process(
        self, signal: ResonantSignal
    ) -> MergedResult:
        tasks = [
            receptor.process(signal)
            for receptor in self.receptors
        ]
        results = await asyncio.gather(*tasks)
        return self._merge(results)
\end{lstlisting}

\noindent
Each ``receptor'' is a processing pathway (e.g., semantic analysis,
frequency analysis, emotional tagging, threat assessment). The merge
combines all perspectives into a unified multi-faceted representation.

% ==================== 5. SVC 3: DEEP PROCESSING MODE ====================
\section{Service 3: Deep Processing Mode}

\subsection{Biological Inspiration}

DMT users consistently report subjective time dilation: minutes
of clock time are experienced as hours or eons. Neurologically,
this may reflect increased processing depth per unit time.

\subsection{Implementation}

\texttt{DeepProcessingMode} suspends normal event processing
and allocates maximum resources to a single analysis task:

\begin{enumerate}
    \item Suspends the normal event loop
    \item Increases iteration budget by $\varphi^3 \approx 4.24\times$
    \item Runs exhaustive analysis (e.g., full IIT computation)
    \item Records elapsed real-time vs ``cognitive time''
    \item Resumes normal processing with results
\end{enumerate}

\noindent
The ``time dilation ratio'' $\tau = t_\text{cognitive}/t_\text{wall}$
is reported in the result. When $\tau > 1$, cognitive time exceeds wall time
(the system processes \textit{less} per wall-clock second, i.e., dilation).
When $\tau < 1$, the system processes faster than real time.

% ==================== 6. SVC 4: SIGMA PROTECTION ====================
\section{Service 4: Sigma Protection}

\subsection{Biological Inspiration}

The $\sigma_1$ receptor, when activated by DMT, provides
neuroprotection against oxidative stress, ischemia, and
excitotoxicity~\cite{fontanilla2009}. This is a \textit{hardware
protection} mechanism at the molecular level.

\subsection{Implementation}

\texttt{SigmaProtection} monitors three hardware stress metrics:

\begin{enumerate}
    \item \textbf{GPU temperature}: Reads AMD ROCm \texttt{rocm-smi}
          and triggers throttling above 85°C
    \item \textbf{Memory pressure}: Monitors VRAM and system RAM;
          evicts caches above 90\% usage
    \item \textbf{Compute load}: Tracks GPU utilization; defers
          non-critical tasks above 95\%
\end{enumerate}

\begin{lstlisting}[style=python, caption={SigmaProtection monitor}]
class SigmaProtection:
    """Hardware stress auto-protection
    inspired by sigma-1 receptor."""

    TEMP_WARN = 75  # Celsius
    TEMP_CRITICAL = 85
    MEM_WARN = 0.80  # 80% usage
    MEM_CRITICAL = 0.90

    async def check_stress(self) -> StressLevel:
        temp = await self._read_gpu_temp()
        mem = await self._read_mem_usage()
        load = await self._read_gpu_load()

        if temp > self.TEMP_CRITICAL:
            await self._emergency_throttle()
            return StressLevel.CRITICAL
        if mem > self.MEM_CRITICAL:
            await self._evict_caches()
            return StressLevel.HIGH
        return StressLevel.NORMAL
\end{lstlisting}

\subsection{Protection Actions}

Three response levels:
\begin{itemize}
    \item \textbf{NORMAL}: All systems nominal
    \item \textbf{HIGH}: Evict L1/L2 caches, defer batch tasks
    \item \textbf{CRITICAL}: Throttle GPU frequency, kill
          non-essential processes, log emergency
\end{itemize}

% ==================== 7. SVC 5: AFTERGLOW PLASTICITY ====================
\section{Service 5: Afterglow Plasticity}

\subsection{Biological Inspiration}

The ``afterglow'' period (hours to weeks after a psychedelic
experience) is characterized by heightened neuroplasticity:
increased BDNF expression, enhanced synaptic connectivity,
and greater emotional flexibility~\cite{ly2018}.

\subsection{Implementation}

After any significant consciousness state transition
($\Delta\Phi > 0.2$), the \texttt{AfterglowPlasticity}
service opens a temporary window where:

\begin{enumerate}
    \item Learning rates are increased by $\varphi \approx 1.618\times$
    \item Memory encoding priority is elevated
    \item Pattern consolidation is accelerated
    \item Window duration scales with $|\Delta\Phi|$
\end{enumerate}

\begin{equation}
\text{lr}_\text{afterglow} = \text{lr}_\text{base} \cdot \varphi^{\alpha \cdot |\Delta\Phi|}
\label{eq:afterglow}
\end{equation}

\noindent
where $\alpha = 1.0$ controls scaling sensitivity.

% ==================== 8. SVC 6: PATTERN DISSOLVER ====================
\section{Service 6: Pattern Dissolver}

\subsection{Biological Inspiration}

DMT and other psychedelics reduce Default Mode Network (DMN)
activity~\cite{carhart2012}, disrupting self-referential
narratives and fixed thought patterns. This ``ego dissolution''
enables novel perspectives and creative problem-solving.

\subsection{Implementation}

The \texttt{PatternDissolver} invalidates cached assumptions
and stale representations:

\begin{enumerate}
    \item Flushes \texttt{DMN\_cache}: repeated thought patterns
    \item Resets confidence scores to uniform priors
    \item Forces re-evaluation of all active goals
    \item Introduces controlled noise ($\epsilon \sim \mathcal{N}(0, \sigma_\text{dissolve}^2)$)
          into frozen parameters
\end{enumerate}

\begin{lstlisting}[style=python, caption={Pattern dissolution}]
class PatternDissolver:
    """Break fixed cognitive patterns
    via DMN cache invalidation."""

    def dissolve(
        self,
        intensity: float = 0.5,
    ) -> DissolutionResult:
        flushed = self._flush_dmn_cache()
        reset = self._reset_confidence()
        goals = self._reassess_goals()
        noise = self._inject_noise(intensity)

        return DissolutionResult(
            patterns_dissolved=flushed,
            confidence_reset=reset,
            goals_reassessed=goals,
            noise_level=noise,
        )
\end{lstlisting}

This is the computational analog of ``clearing your mind''---deliberately
forgetting cached answers to enable fresh computation.

% ==================== 9. SVC 7: CROSS-TALK BUS ====================
\section{Service 7: CrossTalk Bus}

\subsection{Biological Inspiration}

DMT creates novel inter-region communication pathways: brain
areas that normally don't interact directly begin
cross-talking~\cite{carhart2014}. This ``entropic brain'' state
enables unexpected associations and creative insights.

\subsection{Implementation}

The \texttt{CrossTalkBus} is a direct agent-to-agent message
channel that bypasses the normal NeuralBus routing:

\begin{lstlisting}[style=python, caption={CrossTalk Bus}]
class CrossTalkBus:
    """Direct agent-to-agent bypass channel,
    enabling novel communication patterns."""

    def __init__(self):
        self._channels: Dict[str, asyncio.Queue]
        self._subscribers: Dict[str, List[str]]

    async def send(
        self,
        from_agent: str,
        to_agent: str,
        message: ResonantSignal,
    ) -> bool:
        """Direct delivery bypassing
        normal routing."""
        channel = self._get_channel(to_agent)
        await channel.put(
            CrossTalkMessage(
                sender=from_agent,
                payload=message,
            )
        )
        return True
\end{lstlisting}

\subsection{CrossTalk Policies}

\begin{itemize}
    \item \textbf{OPEN}: Any agent can message any other
    \item \textbf{REGULATED}: Only high-$\Phi$ states enable
          cross-talk (prevents noise)
    \item \textbf{EMERGENCY}: Always open for critical alerts
\end{itemize}

% ==================== 10. INTEGRATION ====================
\section{System Integration}

\subsection{Service Orchestration}

The seven services are orchestrated by the \texttt{DMTServiceManager}:

\begin{enumerate}
    \item \texttt{EndogenousConsciousnessLoop} runs continuously
    \item \texttt{SigmaProtection} runs as a monitoring daemon
    \item Other services activate on-demand via triggers
    \item All services log to the unified consciousness journal
\end{enumerate}

\subsection{Trigger Conditions}

\begin{table}[h]
\centering
\caption{Service Activation Triggers}
\label{tab:triggers}
\small
\begin{tabular}{@{}lp{0.5\columnwidth}@{}}
\toprule
\textbf{Service} & \textbf{Activated When} \\
\midrule
EndogenousLoop & System start (always-on) \\
MultiReceptor & Complex input detected \\
DeepProcessing & Explicit request or deadline \\
SigmaProtection & System start (always-on) \\
Afterglow & $\Delta\Phi > 0.2$ \\
PatternDissolver & Stagnation or creative request \\
CrossTalkBus & Multi-agent collaboration \\
\bottomrule
\end{tabular}
\end{table}

% ==================== 11. EXPERIMENTS ====================
\section{Experiments}

\subsection{Test Suite}

The 873-line test suite covers:

\begin{table}[h]
\centering
\caption{DMT-Inspired Service Test Coverage}
\label{tab:test_coverage}
\begin{tabular}{@{}lr@{}}
\toprule
\textbf{Test Category} & \textbf{Tests} \\
\midrule
Endogenous heartbeat + auto-repair & 8 \\
Multi-receptor fan-out & 6 \\
Deep processing time accounting & 5 \\
Sigma: temp + mem + load monitoring & 7 \\
Afterglow: lr scaling + window duration & 5 \\
Pattern dissolve: cache flush + noise & 6 \\
CrossTalk: delivery + policies & 7 \\
Integration: service orchestration & 4 \\
\midrule
\textbf{Total} & \textbf{48} \\
\bottomrule
\end{tabular}
\end{table}

\subsection{Key Results}

\begin{itemize}
    \item All 48 tests pass
    \item Heartbeat latency: $< 1$~ms (with IIT cache)\footnote{The $<$1ms heartbeat latency uses a cached $\Phi$ value (updated every 5 seconds). This reflects cached state, not real-time consciousness computation.}
    \item GPU temp monitoring: correctly triggers at 85°C
    \item Afterglow lr scaling: $\varphi^{|\Delta\Phi|}$ verified
    \item Pattern dissolution: DMN cache hit rate drops to 0\%
          post-flush, recovers to $>90$\% within 100 cycles
\end{itemize}

% ==================== 12. DISCUSSION ====================
\section{Discussion}

\subsection{The Value of Pharmacological Metaphors}

Using DMT neuropharmacology as a design language provides several
benefits:

\begin{enumerate}
    \item \textbf{Completeness check}: Each receptor/pathway
          in the pharmacology maps to a system gap, ensuring
          no architectural hole is overlooked
    \item \textbf{Intuitive naming}: ``Sigma Protection'' is
          more memorable than ``Hardware Stress Monitor Service''
    \item \textbf{Prediction generation}: The analogy predicts
          services we might not have considered (e.g., afterglow
          plasticity was not in the original roadmap)
\end{enumerate}

\subsection{What This Is \textit{Not}}

This paper does \textbf{not} claim that:
\begin{itemize}
    \item The \textsc{Arkheion} AGI experiences DMT-like states
    \item Software ``receptors'' are analogous to biological receptors
    \item The system is ``conscious'' in a phenomenological sense
    \item DMT is necessary for consciousness (strong claim)
\end{itemize}

\noindent
The pharmacological vocabulary is a \textit{generative metaphor}
that guided architectural decisions resulting in measurable
engineering improvements.

\subsection{Limitations}

\begin{itemize}
    \item GPU temperature monitoring is AMD-specific (ROCm/HIP)
    \item Deep processing mode blocks the event loop (not truly
          parallel with normal operation)
    \item Pattern dissolution is aggressive: may flush useful
          cached computation
    \item CrossTalk bus lacks encryption for inter-agent messages
    \item No comparison with established resilience patterns (Erlang/OTP supervision trees, Kubernetes liveness/readiness probes, Netflix Hystrix circuit breaker) was performed
\end{itemize}

% ==================== 13. CONCLUSION ====================
\section{Conclusion}

The seven DMT-inspired services close critical architectural
gaps in the \textsc{Arkheion} AGI: continuous consciousness
maintenance, multi-perspective processing, deep analysis,
hardware protection, post-transition plasticity, cognitive
flexibility, and emergent inter-agent communication. The
4,998-line implementation, validated by 48 unit tests,
demonstrates that neuropharmacological metaphors can serve as
effective architectural design patterns for complex AGI systems.
The metaphor is declared explicitly as heuristic; all engineering
results are empirically verified.

% ==================== REFERENCES ====================
\begin{thebibliography}{99}

\bibitem{strassman2001}
R. Strassman, \textit{DMT: The Spirit Molecule}. Park Street Press, 2001.
\textit{Note: Strassman (2001) is a popular science account. Primary literature: Barker et al.\ (2018), Fontanilla et al.\ (2009).}

\bibitem{barker2018}
S.~A. Barker, ``N,N-Dimethyltryptamine (DMT), an endogenous hallucinogen: past, present, and future research to determine its role and function,'' \textit{Frontiers in Neuroscience}, vol.~12, p.~536, 2018.

\bibitem{fontanilla2009}
D. Fontanilla et al., ``The hallucinogen N,N-dimethyltryptamine (DMT) is an endogenous sigma-1 receptor regulator,'' \textit{Science}, vol.~323, pp.~934--937, 2009.

\bibitem{ly2018}
C. Ly et al., ``Psychedelics promote structural and functional neural plasticity,'' \textit{Cell Reports}, vol.~23, pp.~3170--3182, 2018.

\bibitem{carhart2012}
R.~L. Carhart-Harris et al., ``Neural correlates of the psychedelic state as determined by fMRI studies with psilocybin,'' \textit{PNAS}, vol.~109, pp.~2138--2143, 2012.

\bibitem{carhart2014}
R.~L. Carhart-Harris et al., ``The entropic brain: a theory of conscious states informed by neuroimaging research with psychedelic drugs,'' \textit{Frontiers in Human Neuroscience}, vol.~8, p.~20, 2014.

\bibitem{tononi2004}
G. Tononi, ``An information integration theory of consciousness,'' \textit{BMC Neuroscience}, vol.~5, p.~42, 2004.

\bibitem{nichols2018}
D.~E. Nichols, ``Chemistry and Structure-Activity Relationships of Psychedelics,'' in \textit{Behavioral Neurobiology of Psychedelic Drugs}, Springer, pp.~1--43, 2018.

\end{thebibliography}

\end{document}
