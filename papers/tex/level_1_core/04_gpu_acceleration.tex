%% ARKHEION AGI 2.0 - GPU Acceleration Paper
%% Heterogeneous GPU Optimization with AMD ROCm
%% Author: Jhonatan Vieira Feitosa <ooriginador@gmail.com>
%% Date: February 2026

\documentclass[11pt,twocolumn]{article}

% Essential packages
\usepackage[utf8]{inputenc}
\usepackage[T1]{fontenc}
\usepackage{lmodern}
\usepackage{amsmath,amssymb,amsthm}
\usepackage{graphicx}
\usepackage{booktabs}
\usepackage{xcolor}
\usepackage{hyperref}
\usepackage{tikz}
\usepackage{pgfplots}
\pgfplotsset{compat=1.18}
\usepackage{float}
\usepackage{fancyhdr}
\usepackage{geometry}
\usepackage{caption}

% Page geometry
\geometry{margin=0.75in}

% Tolerance
\tolerance=1000
\emergencystretch=3em
\hyphenpenalty=500

% Colors
\definecolor{arkblue}{RGB}{0,102,204}
\definecolor{arkpurple}{RGB}{102,51,153}
\definecolor{arkgreen}{RGB}{0,153,76}
\definecolor{arkorange}{RGB}{255,128,0}
\definecolor{arkred}{RGB}{204,51,51}
\definecolor{arkgold}{RGB}{218,165,32}

% Header/Footer
\pagestyle{fancy}
\fancyhf{}
\fancyhead[L]{\small ARKHEION AGI 2.0}
\fancyhead[R]{\small GPU Acceleration}
\fancyfoot[C]{\thepage}
\renewcommand{\headrulewidth}{0.4pt}

% Code Listing
\usepackage{listings}
\lstset{
    language=Python,
    basicstyle=\ttfamily\scriptsize,
    keywordstyle=\color{arkblue},
    stringstyle=\color{arkgreen},
    commentstyle=\color{gray}\itshape,
    numbers=none,
    frame=single,
    breaklines=true,
    breakatwhitespace=true,
    postbreak=\mbox{\textcolor{gray}{$\hookrightarrow$}\space},
    columns=flexible,
    keepspaces=true,
    showstringspaces=false,
    backgroundcolor=\color{gray!5}
}

% Hyperref
\hypersetup{
    colorlinks=true,
    linkcolor=arkblue,
    citecolor=arkpurple,
    urlcolor=arkblue
}

% Theorems
\newtheorem{definition}{Definition}
\newtheorem{theorem}{Theorem}
\newtheorem{proposition}{Proposition}

\title{\textbf{Heterogeneous GPU Acceleration}\\[0.3em]
\large ROCm/HIP Optimization for AMD Radeon Hardware}

\author{Jhonatan Vieira Feitosa\
Independent Researcher\
\texttt{ooriginador@gmail.com}\
Manaus, Amazonas, Brazil}

\date{February 2026}

\begin{document}

\maketitle

\begin{abstract}
We present a heterogeneous GPU acceleration system for ARKHEION AGI 2.0 cognitive workloads, optimized for AMD Radeon RX 6600M (gfx1030) hardware with ROCm 6.2. The system achieves \textbf{6.2$\times$--10$\times$ speedup} over CPU baselines on tensor operations, \textbf{224 GB/s memory bandwidth} utilization, and \textbf{28 compute units} parallelism. We implement unified acceleration across CUDA-equivalent HIP kernels, SIMD vectorization, and Smart Access Memory (SAM). The paper distinguishes between "GPU" as hardware reality (empirical) and vendor-specific marketing terms like "infinity cache" (heuristic branding).

\vspace{0.5em}
\noindent\textbf{Keywords:} GPU acceleration, ROCm, HIP, CUDA, parallel computing, AMD, ARKHEION AGI
\end{abstract}

\section*{Epistemological Note}
\textit{This paper distinguishes between \textbf{heuristic} concepts and \textbf{empirical} results.}

\vspace{0.5em}
{\small
\begin{tabular}{@{}p{0.11\columnwidth}p{0.65\columnwidth}@{}}
\textbf{Heuristic:} & Vendor marketing, heterogeneous \\
\textbf{Empirical:} & 28 CUs, 224 GB/s, 6--10$\times$ faster \\
\end{tabular}
}

\vspace{0.5em}
We measure actual hardware performance (bandwidth, throughput, latency), not marketing slogans. Terms like "heterogeneous" describe architectural patterns, not magical properties.

\section{Introduction}

Modern AI workloads demand massive parallel computation:
\begin{itemize}
    \item Neural training: matrix multiply (GEMM) dominates
    \item Quantum simulation: vector operations on $2^n$ states
    \item Holographic encoding: wavelet transforms
    \item Consciousness metrics: entropy calculations
\end{itemize}

GPUs provide 100--1000$\times$ more compute than CPUs for these tasks. However, AMD ROCm ecosystem lags NVIDIA CUDA in tooling maturity, requiring careful optimization.

\textbf{Module Scale:} The GPU acceleration subsystem comprises 4 Python integration files (~2K LOC) plus 67 C++/HIP native source files (~29K LOC), with 22 dedicated test files and 24 Python-accessible GPU functions.

\subsection{Hardware Context}

\textbf{AMD Radeon RX 6600M (Mobile)}
\begin{itemize}
    \item Architecture: RDNA 2 (gfx1030)
    \item Compute Units: 28 (1792 stream processors)
    \item Base/Boost Clock: 2177 / 2382 MHz
    \item Memory: 8GB GDDR6
    \item Memory Bandwidth: 224 GB/s
    \item Peak FP32: 10.8 TFLOPS
    \item TDP: 100W
\end{itemize}

\textbf{Comparison:} NVIDIA RTX 3060 Mobile (similar price):
\begin{itemize}
    \item 3840 CUDA cores, 12GB GDDR6, 360 GB/s, 12.7 TFLOPS
\end{itemize}

AMD offers 67\% memory capacity but 62\% bandwidth of NVIDIA equivalent.

\section{Background}

\subsection{ROCm vs CUDA}

\textbf{ROCm} (Radeon Open Compute) is AMD's GPU compute platform:
\begin{itemize}
    \item HIP: CUDA-like programming model
    \item MIOpen: cuDNN equivalent for deep learning
    \item rocBLAS: cuBLAS equivalent for linear algebra
    \item Open-source toolchain
\end{itemize}

\textbf{PyTorch ROCm:} AMD maintains PyTorch fork with HIP backend.

\subsection{Memory Hierarchy}

\begin{table}[H]
\centering
\caption{GPU Memory Hierarchy}
\small
\begin{tabular}{@{}lrrr@{}}
\toprule
\textbf{Level} & \textbf{Size} & \textbf{BW} & \textbf{Latency} \\
\midrule
Registers & 256KB & -- & 1 cycle \\
L1 Cache & 128KB & 2TB/s & 4 cycles \\
L2 Cache & 4MB & 1TB/s & 40 cycles \\
VRAM & 8GB & 224GB/s & 200 cycles \\
System RAM & 16GB & 25GB/s & 400+ cycles \\
\bottomrule
\end{tabular}
\end{table}

Optimization goal: maximize L1/L2 cache hits, minimize VRAM $\leftrightarrow$ RAM transfers.

\subsection{SIMD Vectorization}

AMD GPUs execute in wavefronts (64-wide SIMD):
\begin{equation}
Throughput = CUs \times Clock \times Ops/Cycle
\end{equation}

For FP32: $28 \times 2.382 \times 64 \approx 4{,}265$ GFLOPS (FP32 theoretical peak).\footnote{AMD officially rates the RX 6600M at 5.40 TFLOPS FP32, considering 1,792 stream processors at $2\times$clock for FMA operations.}

\section{Implementation}

\subsection{Unified Acceleration API}

{\small
\begin{verbatim}
class UnifiedGPUManager:
    def __init__(self):
        self.detect_devices()
        self.select_backend()
        self.allocate_memory()

    def execute(self, kernel, data):
        if rocm_available:
            return self.hip_execute(
                kernel, data)
        elif cuda_available:
            return self.cuda_execute(
                kernel, data)
        else:
            return self.cpu_fallback(
                kernel, data)
\end{verbatim}
}

Automatic backend selection based on availability.

\subsection{Memory Management}

\textbf{Smart Access Memory (SAM):} AMD's resizable BAR technology allowing CPU direct access to full 8GB VRAM.

\textbf{Measured Benefit:}
\begin{itemize}
    \item CPU $\to$ GPU: 12.8 GB/s (SAM) vs 8.5 GB/s (baseline)
    \item +50\% transfer bandwidth
\end{itemize}

\textbf{Implementation:}
{\small
\begin{verbatim}
def transfer_with_sam(data):
    if sam_available:
        # Direct CPU access to VRAM
        vram_ptr = map_vram_to_cpu()
        memcpy(vram_ptr, data, len(data))
    else:
        # Traditional PCIe transfer
        gpu.copy_to_device(data)
\end{verbatim}
}

\subsection{HIP Kernel Example}

Matrix multiplication kernel (simplified):

{\small
\begin{verbatim}
__global__ void matmul_kernel(
    float* A, float* B, float* C,
    int M, int N, int K) {

    int row = hipBlockIdx_y * 16 +
              hipThreadIdx_y;
    int col = hipBlockIdx_x * 16 +
              hipThreadIdx_x;

    float sum = 0.0f;
    for (int k = 0; k < K; ++k) {
        sum += A[row*K + k] * B[k*N + col];
    }
    C[row*N + col] = sum;
}
\end{verbatim}
}

\textbf{Note:} The simplified kernel shown above omits the shared-memory tiling optimization for brevity; the production implementation uses $32 \times 32$ tile blocking.

\noindent Production optimizations applied:
\begin{itemize}
    \item Shared memory tiling ($32\times 32$)
    \item Coalesced memory access
    \item Loop unrolling
\end{itemize}

\section{Experiments}

\subsection{Tensor Operations}

\textbf{Test:} Matrix multiply (4096$\times$4096 FP32)

\begin{table}[H]
\centering
\caption{GEMM Performance}
\small
\begin{tabular}{@{}lrrr@{}}
\toprule
\textbf{Backend} & \textbf{Time} & \textbf{GFLOPS} & \textbf{Speedup} \\
\midrule
CPU (NumPy) & 1.85s & 74 & 1.0$\times$ \\
GPU (rocBLAS) & 0.30s & 458 & 6.2$\times$ \\
Direct Kernel Launch\footnote{Not to be confused with NVIDIA GPUDirect\textsuperscript{\textregistered} or AMD Infinity Fabric Direct, which refer to GPU-to-GPU or GPU-to-NIC RDMA technologies. ``Direct Kernel Launch'' here means bypassing Python overhead to invoke HIP kernels directly.} & 0.18s & 763 & 10.3$\times$ \\
\bottomrule
\end{tabular}
\end{table}

Direct Kernel Launch bypasses Python wrappers for 1.7$\times$ additional gain over rocBLAS.

\subsection{Memory Bandwidth}

\textbf{Test:} Copy 1GB data host $\leftrightarrow$ device

\begin{table}[H]
\centering
\caption{Memory Bandwidth (GB/s)}
\small
\begin{tabular}{@{}lrr@{}}
\toprule
\textbf{Direction} & \textbf{Baseline} & \textbf{SAM} \\
\midrule
Host $\to$ Device & 8.5 & 12.8 \\
Device $\to$ Host & 8.2 & 12.5 \\
Device $\to$ Device & 218 & 224 \\
\bottomrule
\end{tabular}
\end{table}

SAM improves PCIe transfers by 50\%. Intra-device bandwidth near theoretical 224 GB/s.

\subsection{Quantum Simulation}

\textbf{Test:} 16-qubit state vector ($2^{16}$ = 65536 complex)

\begin{table}[H]
\centering
\caption{Quantum Gate Performance}
\small
\begin{tabular}{@{}lrrr@{}}
\toprule
\textbf{Gate} & \textbf{CPU} & \textbf{GPU} & \textbf{Speedup} \\
\midrule
Hadamard & 5.0ms & 0.8ms & 6.2$\times$ \\
CNOT & 7.2ms & 1.1ms & 6.5$\times$ \\
QFT & 45ms & 6.5ms & 6.9$\times$ \\
\bottomrule
\end{tabular}
\end{table}

Consistent 6--7$\times$ speedup on vectorized operations.

\subsection{Neural Network Training}

\textbf{Test:} NeRF model (256$\times$256 resolution)

\begin{table}[H]
\centering
\caption{NeRF Training (100 epochs)}
\small
\begin{tabular}{@{}lrr@{}}
\toprule
\textbf{Metric} & \textbf{CPU} & \textbf{GPU} \\
\midrule
Time/epoch & 42s & 6.8s \\
Total time & 70min & 11.3min \\
VRAM usage & -- & 6.9GB \\
Power (avg) & 45W & 85W \\
\bottomrule
\end{tabular}
\end{table}

6.2$\times$ training speedup at 1.9$\times$ power cost (2.1 J/epoch efficiency).

\subsection{Holographic Compression}

\textbf{Test:} Wavelet transform (4096$\times$4096 image)

\begin{table}[H]
\centering
\caption{Wavelet Transform Performance}
\small
\begin{tabular}{@{}lrr@{}}
\toprule
\textbf{Backend} & \textbf{Time} & \textbf{Speedup} \\
\midrule
CPU (SciPy) & 125ms & 1.0$\times$ \\
GPU (CuPy) & 22ms & 5.7$\times$ \\
\bottomrule
\end{tabular}
\end{table}

\section{Discussion}

\subsection{ROCm Maturity}

\textbf{Strengths:}
\begin{itemize}
    \item Open-source stack
    \item Good PyTorch integration
    \item Improving rapidly
\end{itemize}

\textbf{Weaknesses:}
\begin{itemize}
    \item Installation complexity
    \item Limited framework support vs CUDA
    \item Spotty documentation
    \item Driver stability issues
\end{itemize}

\textbf{Verdict:} ROCm is production-ready for PyTorch workloads but requires expertise.

\subsection{AMD vs NVIDIA}

\textbf{For ARKHEION AGI:}
\begin{itemize}
    \item Quantum sim: Both adequate (6$\times$ speedup)
    \item Neural training: NVIDIA 20--30\% faster
    \item Price: AMD 15\% cheaper (RX 6600M vs RTX 3060)
    \item Open-source: AMD superior
\end{itemize}

\textbf{Decision:} AMD chosen for cost and open ecosystem, accepting performance gap.

\subsection{Optimization Impact}

\begin{table}[H]
\centering
\caption{Cumulative Optimizations}
\small
\begin{tabular}{@{}lr@{}}
\toprule
\textbf{Technique} & \textbf{Gain} \\
\midrule
Baseline GPU & 4.2$\times$ \\
+ Memory coalescing & 5.1$\times$ \\
+ Shared memory & 6.2$\times$ \\
+ Loop unrolling & 7.8$\times$ \\
+ Direct Kernel Launch & 10.0$\times$ \\
\bottomrule
\end{tabular}
\end{table}

Each optimization layer compounds. Final 10$\times$ from 5 techniques.

\section{Limitations}

\begin{enumerate}
    \item \textbf{8GB VRAM:} Limits model size (16-qubit max, 512$\times$512 NeRF)
    \item \textbf{Mobile GPU:} 100W TDP lower than desktop (150W+)
    \item \textbf{ROCm support:} Not all libraries work (e.g., cuDF missing)
    \item \textbf{Driver bugs:} Occasional hangs requiring reboot
    \item \textbf{Windows ROCm:} Experimental, use Linux
\end{enumerate}

\section{Conclusion}

We achieved 6.2--10$\times$ GPU acceleration on AMD RX 6600M (gfx1030) using ROCm 6.2, PyTorch, and custom HIP kernels. Key results:

\begin{itemize}
    \item GEMM: 763 GFLOPS (10$\times$ vs CPU)
    \item Memory BW: 224 GB/s (near theoretical)
    \item Quantum gates: 6.5$\times$ avg speedup
    \item NeRF training: 70min $\to$ 11min
\end{itemize}

\textbf{ROCm Verdict:} Production-ready for PyTorch but requires Linux + expertise.

\textbf{Future Work:} Explore multi-GPU (2$\times$ RX 6600M), tensor cores emulation, and sparse tensor optimization.

\section{References}

\begin{enumerate}
    \item AMD. (2024). \textit{ROCm Documentation}. \url{https://rocm.docs.amd.com}
    \item NVIDIA. (2023). \textit{CUDA C++ Programming Guide}.
    \item Kirk, D. B., \& Hwu, W. W. (2016). \textit{Programming Massively Parallel Processors}. Morgan Kaufmann.
    \item PyTorch Team. (2024). \textit{PyTorch ROCm Support}. \url{https://pytorch.org/get-started/rocm-support/}
    \item Hennessy, J. L., \& Patterson, D. A. (2017). \textit{Computer Architecture: A Quantitative Approach}. 6th ed.
\end{enumerate}

\end{document}
