% ARKHEION AGI 2.0 - Paper 48: Forge Runtime
% Jhonatan Vieira Feitosa | Manaus, Amazonas, Brazil
% February 2026

\documentclass[11pt,twocolumn]{article}

% Encoding and fonts
\usepackage[utf8]{inputenc}
\usepackage[T1]{fontenc}
\usepackage{lmodern}

% Layout
\usepackage[margin=0.75in]{geometry}
\usepackage{fancyhdr}

% Mathematics
\usepackage{amsmath,amssymb}

% Graphics and colors
\usepackage{xcolor}
\usepackage{tikz}
\usetikzlibrary{arrows.meta,shapes,positioning}

% Tables
\usepackage{booktabs}
\usepackage{multirow}

% Code listings
\usepackage{listings}

% Hyperlinks
\usepackage{hyperref}

% ==================== COLORS ====================
\definecolor{arkblue}{RGB}{0,102,204}
\definecolor{arkpurple}{RGB}{102,51,153}
\definecolor{arkgreen}{RGB}{0,153,76}
\definecolor{arkgold}{RGB}{218,165,32}
\definecolor{rustcolor}{RGB}{183,65,14}

% ==================== LISTINGS ====================
\lstset{
    basicstyle=\ttfamily\scriptsize,
    breaklines=true,
    breakatwhitespace=true,
    postbreak=\mbox{\textcolor{gray}{$\hookrightarrow$}\space},
    columns=flexible,
    keepspaces=true,
    showstringspaces=false,
    numbers=none,
    backgroundcolor=\color{gray!5},
    frame=single,
    rulecolor=\color{gray!30}
}

\lstdefinestyle{rust}{
    morekeywords={fn,let,mut,pub,struct,impl,enum,use,mod,self,
                  async,await,Result,Ok,Err,Option,Some,None,
                  trait,where,for,in,match,return,if,else,
                  Vec,String,HashMap,Arc,Mutex,Box}
}

\lstdefinestyle{python}{
    language=Python,
    morekeywords={self,True,False,None,dataclass,Optional,List,Dict}
}

% ==================== HEADER/FOOTER ====================
\pagestyle{fancy}
\fancyhf{}
\fancyhead[L]{\small\textcolor{arkblue}{ARKHEION AGI 2.0}}
\fancyhead[R]{\small Paper 48: Forge Runtime}
\fancyfoot[C]{\thepage}
\renewcommand{\headrulewidth}{0.4pt}

% ==================== HYPERREF ====================
\hypersetup{
    colorlinks=true,
    linkcolor=arkblue,
    urlcolor=arkpurple,
    citecolor=arkgreen,
    pdftitle={Forge: A Rust Runtime for Ternary Model Evolution},
    pdfauthor={Jhonatan Vieira Feitosa}
}

% ==================== TITLE ====================
\title{
    \vspace{-1.5cm}
    {\Large\textbf{Forge: A Rust Runtime for\\Ternary Model Evolution}}\\[0.3em]
    {\large Cross-Language AGI Infrastructure with GPU Training,\\MCP Integration, and Gene Pool Management}\\[0.2em]
    {\normalsize ARKHEION AGI 2.0 --- Paper 48}
}

\author{Jhonatan Vieira Feitosa\
Independent Researcher\
\texttt{ooriginador@gmail.com}\
Manaus, Amazonas, Brazil}

\date{February 2026}

\begin{document}

\maketitle

% ==================== ABSTRACT ====================
\begin{abstract}
We present \textbf{Forge}, a high-performance Rust runtime for the
\textsc{Arkheion} AGI system, comprising 149~Rust source files and
approximately 150,000~lines of code organized into 9~crates. Forge
serves as the \textit{performance-critical backbone} of the AGI
infrastructure, handling ternary model storage (\texttt{.nucleus} format),
GPU-accelerated training via AMD ROCm/HIP, gene pool evolution and
management, cross-language Python bridging, and Model Context Protocol
(MCP) server integration exposing 65+ tools for AI-assisted model
engineering. The system achieves memory-safe, zero-cost abstraction
performance through Rust's ownership model, while maintaining
ergonomic interoperability with the 603K-line Python codebase through
\texttt{forge-python} and \texttt{forge-bridge} crates. Key innovations
include:\ (1)~a ternary-native gene representation where each model
parameter is a trit $\in \{-1, 0, +1\}$; (2)~holographic compression
encoding for gene pool storage; (3)~GPU kernel dispatch for training
with AMD RDNA2 hardware; and (4)~a single source of truth (PHI SSOT)
for the golden ratio constant across all crates.

\textbf{Keywords:} Rust, AGI runtime, ternary models, gene pools,
GPU training, MCP, cross-language, ROCm, HIP, gene evolution
\end{abstract}

% ==================== EPISTEMOLOGICAL NOTE ====================
\section*{Epistemological Note}

\textit{This paper documents an implemented and tested system.
All metrics are measured from the actual codebase.}

\vspace{0.3em}
\noindent
\begin{tabular}{@{}p{0.45\columnwidth}p{0.45\columnwidth}@{}}
\textbf{Heuristic:} & \textbf{Empirical:} \\
\footnotesize ``Gene pool'' metaphor & \footnotesize 149 files, $\sim$150K LOC \\
\footnotesize ``Evolution'' terminology & \footnotesize 946 tests passing \\
\footnotesize ``Brain'' for inference & \footnotesize 65+ MCP tools \\
\footnotesize ``Forge'' as crafting metaphor & \footnotesize GPU training verified \\
\end{tabular}

% ==================== 1. INTRODUCTION ====================
\section{Introduction}

The \textsc{Arkheion} AGI system's Python codebase (603K~LOC,
1,827~files) provides flexibility, rapid prototyping, and
compatibility with the PyTorch/NumPy ecosystem. However,
three requirements demand systems-level performance:

\begin{enumerate}
    \item \textbf{Ternary model storage}: Packing trits
          ($\{-1, 0, +1\}$) efficiently requires bit-level
          manipulation inappropriate for Python
    \item \textbf{GPU training}: Kernel dispatch, memory
          management, and training loops benefit from low-level
          hardware control
    \item \textbf{Long-running services}: The MCP server must
          run continuously with minimal memory overhead
\end{enumerate}

\noindent
Forge addresses all three as a Rust workspace with 9 specialized
crates, interoperating with Python through a bridge layer.

\subsection{Contributions}

\begin{enumerate}
    \item 9-crate Rust workspace architecture
    \item \texttt{.nucleus} format for ternary model storage
    \item GPU training pipeline via AMD ROCm/HIP
    \item 65+ MCP tools for AI-assisted model engineering
    \item Cross-language Python bridge
    \item Gene pool evolution with mutation and selection
    \item Holographic compression for gene storage
    \item $\sim$150K LOC with 946 tests
\end{enumerate}

% ==================== 2. CRATE ARCHITECTURE ====================
\section{Crate Architecture}

\begin{table}[h]
\centering
\caption{Forge Workspace Crate Architecture}
\label{tab:crates}
\small
\begin{tabular}{@{}llr@{}}
\toprule
\textbf{Crate} & \textbf{Responsibility} & \textbf{Files} \\
\midrule
\texttt{forge-core} & Data types, trit representation & 18 \\
\texttt{forge-intel} & Gene analysis, diagnostics & 24 \\
\texttt{forge-brain} & Inference, text generation & 16 \\
\texttt{forge-bank} & Persistent gene storage & 14 \\
\texttt{forge-gpu} & ROCm/HIP training kernels & 12 \\
\texttt{forge-mcp} & MCP server, 65+ tools & 22 \\
\texttt{forge-bridge} & Python$\leftrightarrow$Rust interop & 11 \\
\texttt{forge-python} & PyO3 bindings & 8 \\
\texttt{forge-ui} & Terminal UI, progress bars & 10 \\
\midrule
Other (tests, examples) & & 14 \\
\midrule
\textbf{Total} & & \textbf{149} \\
\bottomrule
\end{tabular}
\end{table}

\subsection{Dependency Graph}

\begin{verbatim}
forge-mcp --> forge-intel --> forge-core
          |-> forge-brain --> forge-core
          |-> forge-bank  --> forge-core
          |-> forge-gpu   --> forge-core
forge-bridge --> forge-intel
             |-> forge-brain
forge-python --> forge-bridge
forge-ui --> forge-intel, forge-bank
\end{verbatim}

\noindent
All crates depend on \texttt{forge-core} for fundamental types.

% ==================== 3. FORGE-CORE ====================
\section{forge-core: Ternary Fundamentals}

\subsection{Trit Representation}

The fundamental unit is the \textit{trit} (ternary digit):

\begin{lstlisting}[style=rust, caption={Trit type}]
#[derive(Clone, Copy, PartialEq)]
pub enum Trit {
    Neg = -1,  // -1
    Zero = 0,  //  0
    Pos = 1,   // +1
}
\end{lstlisting}

A \textit{gene} is a named array of trits representing a model
parameter (weight matrix, bias vector, etc.):

\begin{lstlisting}[style=rust, caption={Gene structure}]
pub struct Gene {
    pub id: String,
    pub trits: Vec<Trit>,
    pub shape: Vec<usize>,
    pub domain: String,
    pub quality: f64,  // phi score
}
\end{lstlisting}

\subsection{.nucleus Format}

The \texttt{.nucleus} file format stores gene pools:

\begin{enumerate}
    \item \textbf{Header}: Magic bytes, version, gene count, metadata
    \item \textbf{Gene table}: Sorted index of gene IDs, shapes, offsets
    \item \textbf{Trit data}: Packed trits (5 trits per byte,
          practical; $3^5 = 243 \leq 256$), with a theoretical
          maximum of $\lfloor\log_3 256\rfloor = 5$ complete trits
    \item \textbf{Checksum}: SHA-256 integrity verification
\end{enumerate}

\subsection{PHI Single Source of Truth}

The golden ratio constant is defined exactly once:

\begin{lstlisting}[style=rust, caption={PHI SSOT}]
pub const PHI: f64 = 1.618033988749895;
pub const INV_PHI: f64 = 0.618033988749895;
pub const PHI_SQ: f64 = 2.618033988749895;
\end{lstlisting}

All crates reference \texttt{forge\_core::PHI} ensuring
consistency across the entire Rust codebase.

% ==================== 4. FORGE-INTEL ====================
\section{forge-intel: Gene Intelligence}

The intelligence crate provides diagnostic and analytical
tools for gene pools:

\begin{itemize}
    \item \textbf{Diagnose}: Identify weak genes, dead genes
          (all zeros), and quality outliers
    \item \textbf{A/B Compare}: Compare two gene pools across
          quality metrics
    \item \textbf{Mutation History}: Track evolution lineage
    \item \textbf{Quality Metrics}: $\varphi$-score, entropy,
          trit distribution statistics
    \item \textbf{Auto-Clean}: Remove dead/duplicate genes
\end{itemize}

\subsection{$\varphi$-Score}

The quality of a gene is measured by its $\varphi$-score, computed from
three sacred-geometry-inspired components:

\begin{equation}
\varphi\text{-score}(g) = 0.40 \cdot A_\text{golden}
  + 0.30 \cdot S_\text{entropy}
  + 0.30 \cdot F_\text{fib}
\label{eq:phi_score}
\end{equation}

\noindent
where:
\begin{itemize}
  \item \textbf{Golden alignment}
        $A_\text{golden} = \frac{1}{1 + \min(|r - \varphi|,\;|r - 1/\varphi|)}$,
        with $r = t_+ / t_-$, measures how close the positive/negative
        trit ratio is to $\varphi$ or $1/\varphi$.
  \item \textbf{Entropy score}
        $S_\text{entropy} = -\sum_{v \in \{-1,0,+1\}} p_v \ln p_v \;/\; \ln 3$
        is the normalized Shannon entropy over the trit distribution,
        maximized when trits are evenly balanced.
  \item \textbf{Fibonacci correlation}
        $F_\text{fib}$ is the mean autocorrelation of the trit sequence
        at Fibonacci lag distances ($1, 2, 3, 5, 8, \ldots$), normalized
        to $[0, 1]$, measuring self-similarity at
        Fibonacci-separated positions.
\end{itemize}

\noindent
The score ranges roughly over $[0,\; {\sim}1.6]$; higher values
indicate better structural quality.  This is a \textit{heuristic}
quality metric inspired by sacred geometry, \textbf{not} an IIT~$\Phi$
calculation.

% ==================== 5. FORGE-GPU ====================
\section{forge-gpu: GPU-Accelerated Training}

\subsection{AMD ROCm Integration}

Forge dispatches GPU kernels via AMD ROCm/HIP, targeting
the RDNA2 architecture (gfx1030, AMD Radeon RX 6600M):

\begin{lstlisting}[style=rust, caption={GPU training dispatch}]
pub async fn train_epoch(
    pool: &mut GenePool,
    dataset: &Dataset,
    config: &TrainConfig,
) -> Result<EpochResult> {
    let device = hip::Device::default()?;
    let stream = device.create_stream()?;

    for batch in dataset.batches(config.batch_size) {
        let loss = forward_backward(
            &pool, &batch, &stream
        )?;
        apply_mutations(pool, &loss, config.lr)?;
    }

    Ok(EpochResult {
        loss: total_loss / n_batches,
        mutations: mutation_count,
    })
}
\end{lstlisting}

\subsection{Training Pipeline}

The \texttt{training\_gpu.rs} module (982~LOC) implements:
\begin{enumerate}
    \item Forward pass: ternary matrix multiplication
    \item Loss computation: cross-entropy with label smoothing
    \item Backward pass: gradient approximation for ternary weights
    \item Mutation: probabilistic trit flips guided by gradients
    \item Checkpoint: periodic \texttt{.nucleus} file saves
\end{enumerate}

% ==================== 6. FORGE-MCP ====================
\section{forge-mcp: Model Context Protocol Server}

\subsection{MCP Architecture}

Forge implements a JSON-RPC 2.0 server following the Model
Context Protocol (MCP) specification, enabling AI agents
(GitHub Copilot, Claude, etc.) to interact with gene pools
through natural language:

\begin{lstlisting}[style=rust, caption={MCP tool example}]
#[mcp_tool(
    name = "forge_diagnose",
    description = "Diagnose gene pool health"
)]
pub async fn diagnose(
    file: String,
    domain: Option<String>,
) -> Result<DiagnosticReport> {
    let pool = load_pool(&file)?;
    intel::diagnose(&pool, domain.as_deref())
}
\end{lstlisting}

\subsection{Tool Categories}

65+ MCP tools organized by function:

\begin{table}[h]
\centering
\caption{MCP Tool Categories}
\label{tab:mcp}
\small
\begin{tabular}{@{}lr@{}}
\toprule
\textbf{Category} & \textbf{Tools} \\
\midrule
Gene pool analysis & 12 \\
Gene bank management & 8 \\
Gene evolution & 7 \\
Brain (inference) & 6 \\
GPU training & 5 \\
Model conversion & 4 \\
Compression & 4 \\
Health monitoring & 5 \\
Benchmarking & 3 \\
Data export & 4 \\
Mutation/pruning & 4 \\
Multi-model ops & 3 \\
\midrule
\textbf{Total} & \textbf{65+} \\
\bottomrule
\end{tabular}
\end{table}

% ==================== 7. FORGE-BRAIN ====================
\section{forge-brain: Inference Engine}

The brain crate enables text generation from ternary models:

\begin{enumerate}
    \item \textbf{Tokenization}: UTF-8 byte-pair encoding
    \item \textbf{Forward pass}: Ternary matrix-vector products
          (only additions, no multiplications---since weights
          are $\{-1, 0, +1\}$)
    \item \textbf{Sampling}: Temperature-scaled softmax with
          top-$k$/top-$p$ filtering
    \item \textbf{Deliberation}: Multi-revision reasoning
          (plan $\to$ reflect $\to$ refine)
\end{enumerate}

\subsection{Ternary Efficiency}

With weights $w \in \{-1, 0, +1\}$, the matrix-vector product
$y = Wx$ reduces to:

\begin{equation}
y_i = \sum_{j: w_{ij}=1} x_j - \sum_{j: w_{ij}=-1} x_j
\label{eq:ternary_matmul}
\end{equation}

\noindent
This requires only additions and subtractions---no floating-point
multiplications---enabling efficient inference on CPUs without
dedicated tensor hardware.

% ==================== 8. CROSS-LANGUAGE BRIDGE ====================
\section{Cross-Language Bridge}

\subsection{forge-bridge}

The bridge crate provides Rust functions callable from both
the MCP server and the Python codebase:

\begin{lstlisting}[style=rust, caption={Bridge interface}]
pub trait ForgeBridge: Send + Sync {
    fn load_pool(&self, path: &str)
        -> Result<GenePool>;
    fn diagnose(&self, pool: &GenePool)
        -> Result<Report>;
    fn evolve(&self, pool: &mut GenePool,
        config: &EvolveConfig)
        -> Result<EvolveResult>;
}
\end{lstlisting}

\subsection{forge-python (PyO3)}

The Python crate wraps bridge functions as a native Python
module using PyO3:

\begin{lstlisting}[style=python, caption={Python usage}]
import forge_python as forge

pool = forge.load_pool("model.nucleus")
report = forge.diagnose(pool)
print(f"Quality: {report.phi_score:.3f}")
\end{lstlisting}

% ==================== 9. EVOLUTION ENGINE ====================
\section{Gene Pool Evolution}

\subsection{Mutation Operators}

\begin{itemize}
    \item \textbf{Random flip}: Change a random trit
    \item \textbf{Guided flip}: Flip trits where gradient
          magnitude is highest
    \item \textbf{Cross-over}: Combine trits from two parents
    \item \textbf{Amputate}: Zero out entire genes (destructive)
    \item \textbf{Prune}: Remove low-quality genes
\end{itemize}

\subsection{Selection}

Tournament selection with elitism: top 10\% of genes survive
unchanged, remaining 90\% undergo mutation and selection.

\subsection{Multi-Objective Pareto Evolution}

The system supports Pareto-optimal evolution across objectives:
quality ($\varphi$-score), size (parameter count), and inference
speed (tokens/second).

% ==================== 10. EXPERIMENTS ====================
\section{Experiments}

\subsection{Test Suite}

\begin{table}[h]
\centering
\caption{Forge Test Summary}
\label{tab:tests}
\begin{tabular}{@{}lr@{}}
\toprule
\textbf{Crate} & \textbf{Tests} \\
\midrule
forge-core & 142 \\
forge-intel & 198 \\
forge-brain & 87 \\
forge-bank & 94 \\
forge-gpu & 56 \\
forge-mcp & 134 \\
forge-bridge & 78 \\
forge-python & 42 \\
forge-ui & 31 \\
Integration & 84 \\
\midrule
\textbf{Total} & \textbf{946} \\
\bottomrule
\end{tabular}
\end{table}

\subsection{Performance}

\begin{itemize}
    \item Ternary matmul: $5.2\times$ faster than equivalent
          float32 matmul on CPU (additions only).\footnote{Measured
          on AMD Ryzen~5 with $1024\times 1024$ matrices comparing
          optimized lookup-table ternary matmul against naive
          floating-point multiplication, single-threaded.}
    \item .nucleus load time: $<100$~ms for 268M parameter model
    \item MCP server memory: $<50$~MB idle, $<500$~MB during
          gene pool operations
    \item GPU training throughput: $\sim$2K tokens/s on
          AMD RX 6600M
\end{itemize}

% ==================== 11. DISCUSSION ====================
\section{Discussion}

\subsection{Why Rust?}

\begin{enumerate}
    \item \textbf{Memory safety}: No null pointers, data races,
          or buffer overflows---critical for long-running services
    \item \textbf{Zero-cost abstractions}: Trait-based polymorphism
          with no runtime overhead
    \item \textbf{Cross-compilation}: Single binary deployment
          without Python runtime
    \item \textbf{Ecosystem}: cargo, crates.io, and excellent
          C/C++ interop for GPU libraries
\end{enumerate}

\subsection{Limitations}

\begin{itemize}
    \item AMD ROCm support is less mature than CUDA
    \item PyO3 bridge adds complexity compared to pure Python
    \item 150K LOC Rust codebase maintained by 1 developer + AI
    \item Gene evolution is CPU-bound; GPU evolution planned
\end{itemize}

% ==================== 12. CONCLUSION ====================
\section{Conclusion}

Forge provides the performance-critical runtime layer for the
\textsc{Arkheion} AGI system. Its 9-crate Rust workspace with
$\sim$150K LOC and 946 tests delivers ternary model management,
GPU training, gene evolution, and AI-accessible tooling through
65+ MCP tools. The cross-language bridge enables seamless
interoperation with the 603K-line Python codebase. Forge
demonstrates that Rust's ownership model, zero-cost abstractions,
and memory safety guarantees are well-suited for AGI infrastructure
where reliability and performance coexist.

% ==================== REFERENCES ====================
\begin{thebibliography}{99}

\bibitem{matsakis2014}
N.~D. Matsakis and F.~S. Klock~II, ``The Rust language,'' in \textit{Proc.\ ACM SIGAda}, pp.~103--104, 2014.

\bibitem{klabnik2019}
S. Klabnik and C. Nichols, \textit{The Rust Programming Language}. No Starch Press, 2019.

\bibitem{li2016}
F. Li et al., ``Ternary weight networks,'' in \textit{NeurIPS} Workshop, 2016.

\bibitem{hubara2017}
I. Hubara et al., ``Quantized neural networks: training neural networks with low precision weights and activations,'' \textit{JMLR}, vol.~18, pp.~1--30, 2018.

\bibitem{mcp2024}
Anthropic, ``Model Context Protocol specification,'' 2024. [Online]. Available: \url{https://modelcontextprotocol.io}

\bibitem{rocm2024}
AMD, ``ROCm: Radeon Open Compute Platform,'' 2024. [Online]. Available: \url{https://rocm.docs.amd.com}

\bibitem{pyo3}
PyO3 Contributors, ``PyO3: Rust bindings for the Python interpreter,'' 2024. [Online]. Available: \url{https://pyo3.rs}

\end{thebibliography}

\end{document}
